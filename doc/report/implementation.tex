\chapter{Implementation and optimisation}

We will now consider some more practical issues regarding the actual
implementation and optimisation of the code in the project.

In \GAP, a very fast implementation of the Schreier-Sims algorithm already
exists. It is described in \cite{seress03} and is a heuristic based on
the algorithm described in \cite{seress91}. As with many other
algorithms in \GAP, it works only with permutation groups, so for a
general group $G$ it first calculates a faithful permutation
representation, ie an injective homomorphism $\lambda : G \to
\Sym{X}$, for some finite set $X$, which exists due to the well-known theorem by Cayley. It
then work with the image $\lambda(G)$, or rather with $H \leq S_{\abs{X}}$ such that $H \cong \lambda(G)$.

Maybe the most important motivation for the \GAP~package developed in
this project is the idea that if $G$ is a matrix group, and one work
with the matrices directly instead of first converting to
permutations, then perhaps one could use the additional structure, that
is otherwise thrown away, and come up with a faster algorithm. The code is 
therefore created with the basic assumption that all group elements are
invertible matrices over a finite field, and it uses the internal
\GAP~representation of such matrices.

\section{Schreier tree creation}
An important issue is how the Schreier trees are managed. Actually,
what we really want is the transversals defined by the Schreier trees,
so a possibility is to store not a tree, but just a list of all points
and for each point the element that moves the root to this point. This
can be realised as a tree with height 1, where the edge labels are not
the generators, as in the case of a Schreier tree, but the group
elements, so in general this will take up more memory than the
Schreier tree. On the other hand, to find a coset representative in a
Schreier tree we need to follow a path from a point to the root, which
takes logarithmic time on average, and may take linear time if the
tree is not properly balanced, but in the case where we store the
coset representatives as edge labels, then of course it takes constant
time to find them later. 

To conclude, we have a trade-off between time and space, and in this
project, both of the above strategies are available to the user of the
package. The concrete representation of the trees are as hash tables,
where the keys are the points and the values are the edge labels of
the unique edges directed towards the root, ie we use
''back-pointers''. This makes it possible to perform in constant time
the common task of checking if a given point is in the orbit defined
by a Schreier tree, since this is just a hash table lookup.

\subsection{Extending vs creating}

If we consider \ref{alg:ss} in more detail, then we notice that the
sets $S^i$ are only augmented if they are changed. Therefore, at line
\ref{alg:ss_comp_tree} we could save the tree that is computed and, at
the next time we arrive there, only extend the tree using the new
generators, if there are any.

This may or may not give a faster algorithm. Evidently, there will be
less work in \ref{alg:orbit} to extend a Schreier tree than to create
a new one, but it may happen that the tree becomes more balanced if it
is recomputed using all generators than if it is extended. If the tree
is not balanced, then \ref{alg:trace} will take more time, and this is
in fact the function where most time is spent. In \cite{seress03} it
is claimed that this problem with unbalanced trees are more common for
matrix groups, and empirical studies in this project has shown that in
most cases, it is better to recompute the Schreier trees every time.

\subsection{Shallow trees}
There are algorithms for the creation of Schreier trees that are
guaranteed to make the Schreier trees \emph{shallow}, ie balanced, so
that the they have worst-case logarithmic height. This is crucial if
one wants to have good worst-case complexity, and two algorithms are
described in \cite{seress03}, one deterministic and one probabilistic.
In this project the deterministic algorithm have been implemented,
which is also described in \cite{seress91}. It is too complicated to
be included here in more detail, but the essential idea is to choose a
different set of edge labels, rather than the given generators.

\section{Orbit sizes}
As have been noted earlier, using the Schreier-Sims algorithm for a
matrix group $G \leq \GL(d, q)$ is in a sense doomed from the
beginning, since the complexity is exponential in $d$ when $G$ acts on
$\F_q^d$. One manifestation of doom in this case is that the orbits
may become huge, something that does not happen for permutation
groups. This will make our Schreier trees huge and a large amount of Schreier generators must be created, so the algorithm will be slow.

\subsection{Alternating actions}
To avoid large orbits, one can use another action of $G$. However,
only if the action is faithful is it a permutation representation of
$G$, and this is needed if the Schreier-Sims algoritm is going to
work, otherwise we will calculate a base and strong generating set for the quotient of $G$ with the kernel of the action.

In \cite{butler76}, a clever trick was introduced where base points are
chosen alternatingly as one-dimensional subspaces (lines) and vectors
from those lines. When a vector $v = (v_1, \dotsc, v_d) \in \F_q^d$ is
chosen as base point, it is preceded in the base by the line
$\gen{v}$. The action of $G$ on the lines is known as the
\emph{projective action} and of course it is not faithful, but since the next
point $v$ is from the kernel this does not matter. Also, the subspace
$\gen{v}$ has a canonical representative $(1, v_2 v_1^{-1}, \dotsc,
v_d v_1^{-1})$, which is trivial to find from $v$. Therefore, in terms
of time, the projective action is not particularly more expensive than the action on points.

Now, if $k = \abs{v^G}$ then $\abs{\gen{v}^G} \abs{v^{G_{\gen{v}}}} = k$ and $G_{\gen{v}}$
is the stabiliser of the line containing $v$. This implies that if $u \in
v^{G_{\gen{v}}}$ then $u$ is also on that line, so $u = mv$ where $m
\in \F_q$. We see that $M = \set{m \in \F_q \mid mv \in v^{G_{\gen{v}}}}$ is a
subgroup of $\F_q^{*}$ and thus $l = \abs{v^{G_{\gen{v}}}} = \abs{M}$ divides $q - 1$.

Instead of one orbit of size $k$ we therefore have two orbits of size
$k/l$ and $l$, and since $k \geq k/l + l$ whenever $l \geq 2$ we will for
example need to compute and sift fewer Schreier generators. On the
other hand, our base will probably be longer when using this trick,
and empirical studies in this project has shown that it is not always
a good idea, but it nevertheless implemented and can be used.

\subsection{Eigenspaces}
Another method for producing smaller orbits is described in \cite{murray95}. The idea is to choose the base points to be eigenvectors of the given generator matrices.

Recall from linear algebra that the \emph{characteristic polynomial}
of a matrix $A \in \GL (d, q)$ is $c_A(x) = \det (xI - A)$ where $I$
is the identity matrix, and an \emph{eigenvalue} of $A$ is a root of
$c_A(x)$.  Normally, and \emph{eigenvector} is defined as a vector $v
\in \F_q^d$ such that $v^A = \lambda v$ where $\lambda$ is an
eigenvalue of $A$, or equivalently $v^{g(A)} = 0$ where $g(x)$ is a
linear factor of $c_A(x)$. Here we use a more general version of the
latter definition, where we allow factors of higher degree as well as
linear factors of $c_A(x)$.

We see that for a linear factor $g(x) = x - \lambda$ and a
corresponding eigenvector $v$, the size of the orbit $v^{\gen{A}}$ is
a divisor of $q - 1$, since if $u \in v^{\gen{A}}$ then $u = v^{\alpha
  A} = \alpha \lambda v$, and as before the possible values of
$\alpha$ form a subgroup of $\F_q^{*}$. More generally, for a factor
of $c_A(x)$ of degree $m$, the size of the orbit $v^{\gen{A}}$ for an
eigenvector $v$ is bounded above by $q^m - 1$. To get the smallest
possible orbits, we should therefore choose eigenvectors corresponding
to factors of as small degree as possible, and linear factors are easy
to find since they correspond to eigenvalues, which are easy to
compute.

Note that we have only considered the orbits of groups generated by
single matrices, and the orbit of a matrix group $G = \gen{S}$ need
not be small just because we choose as base point an eigenvector of
one of the generators in $S$, but if we choose a base point that is an
eigenvector of several generators, then it turns out that the orbit
size are more often small. In \cite{murray95} these issues are
investigated and experimented with in some detail, and a heuristic for
finding base points that will hopefully give small orbits is
developed. This has also been implemented in \GAP, and in this project
it is possible to use that algorithm. However, it is not always a good
idea to use it, since there is some overhead, and it is not certain
that the orbit sizes will actually be smaller.

\section{Further developments}

The package contains implementations of two more advanced algorithms,
the so-called \emph{Verify} routine by Sims, which is an algorithm for
verifying if a proposed base and SGS are complete, and the nearly
linear time algorithm for finding a base and an SGS. The complete
descriptions of these algorithms are beyond the scope of this report,
but for completeness we include brief accounts on them.

\subsection{The Verify routine}

This algorithm is due to Sims, and has never been published, but it is
described in \cite{seress03}. Given a group $G = \gen{S}$ acting on
the finite set $X$, a point $\alpha \in X$ and a subgroup $H = \gen{S^{\prime}} \leq G_{\alpha}$, it
checks whether $H = G_{\alpha}$. If this is not the case, the
algorithm computes $g \in G_{\alpha} \setminus H$ to witness that. To
check a whole proposed base and SGS, one then uses the third case of
Theorem \ref{thm_leon} and checks each level.

The algorithm is quite involved, both theoretically and when it comes
to implementing it. It has a recursive nature, inducting on $S \setminus
S^{\prime}$, and it involves things like changing base points and
computing block systems.

\subsection{The nearly linear time algorithm}
For permutation groups $G$ acting on $X$, there is an algorithm for
computing a base and an SGS that runs in nearly linear time, ie linear
in $\abs{X}$ except for some logarithmic factor of $\abs{X}$ and
$\abs{G}$. It is the best known algorithm in terms of time complexity,
and is described in \cite{seress91} as well as in \cite{seress03} and it
is probabilistic. The complexity is achieved by using shallow
Schreier trees, a fast probabilistic verification algorithm, and by
sifting only a few randomly selected Schreier generators. Otherwise,
the algorithm is quite similar to the algorithms we have described
earlier for computing a base and an SGS.

The random Schreier generators are computed using \emph{random
  subproducts} which are described in \cite{seress03} as well as by
selecting random group elements, and the algorithm relies on some
theorems relating the number of Schreier generators to compute to the
given probability of correctness.

