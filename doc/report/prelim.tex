\chapter{Preliminaries}
The definitions and statements in this section are assumed to be
known, but we state them anyway, since authors often use different
notation and sometimes put slightly different meaning to some of the
following concepts (eg. the exact definition of graphs tend to
vary). 

\section{Graph theory}
First some graph theory, where we follow \cite{biggs89}.

\begin{deff} \label{def_graph}
A \emph{directed graph} is an ordered pair $G = (V, E)$ where $V$ is a finite non-empty set, the \emph{vertices} of $G$ and $E \subseteq V \times V$ is the \emph{edges} of $G$.
\end{deff}
\begin{remark}
  A graph in the sense of \ref{def_graph} is sometimes called a
  \emph{combinatorial graph} in the literature, to emphasize that they
  are not \emph{metric graphs} in the sense of \cite{bridson99}. As we
  are interested only in finite graphs and not in geometry, we do not
  make use of this nomenclature.
\end{remark}

\begin{remark}
The definition implies that our graphs have no multiple edges, but may have loops. We will henceforth omit the word ''directed'' since these are the only
graphs we are interested in.
\end{remark}

\begin{deff} \label{def_graph_misc}
Let $G = (V, E)$ is a graph, a sequence $v = v_1, v_2, \dotsc, v_n =
  u$ of vertices of $G$ such that $(v_i, v_{i + 1}) \in E$ for $i = 1,
  \dotsc, n - 1$ is called a \emph{walk} from $v$ to $u$. If $v = u$
  then the walk is called a \emph{cycle}. If all vertices in the walk are
  distinct, it is called a \emph{path}. If every pair of distinct vertices in $G$ can be joined by a path, then $G$ is \emph{connected}.
\end{deff}

\begin{deff}
A graph $G^{\prime} = (V^{\prime}, E^{\prime})$ is a \emph{subgraph} of a graph $G = (V, E)$ if $V^{\prime} \subseteq V$ and $E^{\prime} \subseteq E$.
\end{deff}

\begin{deff} \label{def_tree}
A \emph{tree} is a connected graph without cycles. A \emph{rooted tree} is a tree $T = (V, E)$ where a vertex $r \in V$ have been specified as the \emph{root}.
\end{deff}

\begin{deff}
If $G = (V, E)$ is a graph, then a tree $T$ is a \emph{spanning tree} of $G$ if $T$ is a subgraph of $G$ and $T$ have vertex set $V$.
\end{deff}

The following are elementary and the proofs are omitted.
\begin{pr}
In a tree $T = (V, E)$ we have $\abs{E} = \abs{V} - 1$ and there
exists a \emph{unique} path between every pair of distinct
vertices. Conversely, if $G$ is a graph where every pair of distinct
vertices can be joined by a unique path, then $G$ is a tree.
\end{pr}

\begin{pr}
Every graph has a spanning tree.
\end{pr}

\begin{deff}
If $G = (V, E)$ is a graph, then a \emph{labelling} of $G$ is a function $w : E \to L$ where $L$ is some set of ''labels''. A \emph{labelled graph} is a graph with a corresponding labelling.
\end{deff}

\section{Group theory}
Now some group theory, where we follow our standard references \cite{bb96} and \cite{rose78}.
\begin{note}
All groups in this report are assumed to be finite.
\end{note}

\begin{deff} \label{def_cayley}
If $G = \gen{S}$ is a group, then the \emph{Cayley graph} $\C_G(S)$ is the graph with vertex set $G$ and edges $E = \set{ (g, sg) \mid g \in G, s \in S}$.
\end{deff}

\begin{deff}
An \emph{action} of a group $G$ on a finite set $X$ is a homomorphism
$\lambda : G \to \Sym{X}$ (where $\Sym{X}$ is the group of
permutations on $X$). If $\lambda$ is injective, the action is
\emph{faithful}. 
\end{deff}
\begin{remark}
Following a convention in computational group theory, actions are from the right, and $\lambda(g)x$ is abbreviated with $x^g$, for $g \in G$ and $x \in X$. The elements of $X$ are called \emph{points}. Note that some of the rules for exponents hold since we have an action, eg $(x^g)^h = x^{gh}$.
\end{remark}

\begin{deff}
Let $G$ be a group acting on the finite set $X$. For each point
$\alpha \in X$, the \emph{orbit} of $\alpha$ is $\alpha^G = \set{\beta
\in X \mid \beta = \alpha^g, g \in G}$ and the \emph{stabiliser} of
$\alpha$ is $G_{\alpha} = \set{g \in G \mid \alpha^g = \alpha}$.
\end{deff}

The following are elementary and the proofs are omitted.
\begin{pr}
Let $G$ be a group acting on the finite set $X$. For each $p \in X$ we have $G_p \leq G$, and so we can define inductively 
\begin{equation}
G_{\alpha_1, \alpha_2, \dotsc, \alpha_n} = (G_{\alpha_1, \alpha_2, \dotsc, \alpha_{n - 1}})_{\alpha_n}
\end{equation}
where $n > 1$ and $\alpha_1, \dotsc, \alpha_n \in X$.
\end{pr}

\begin{pr} \label{thm_orbit_stab}
Let $G$ be a group acting on the finite set $X$. For each $p \in X$, the map $\mu_p : G / G_p \to p^G$ given by
\begin{equation}
G_p g \mapsto p^g
\end{equation}
for each $g \in G$, is a bijection. In particular, $\abs{p^G} = [G : G_p]$.
\end{pr}

\section{Computer science}
When it comes to computer science, our standard reference is
\cite{clr90} where all basic computer science notions can be
found. First of all, complexity analysis of algorithms and its
asymptotic notation, in particular the $O$-notation, is assumed to be
known. Basic graph algorithms like breadth-first search, computation
of connected compontents and spanning tree algorithms are also assumed to be known.

Hash tables will likewise be used without further explanation. Even
though the code in the project does not use hashing explicitly, but
rely on \GAP~for that, it is worth mentioning that, to the author's
knowledge, the best general-purpose hash function known to humanity is
described in \cite{jenkins97}.
