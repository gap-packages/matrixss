\chapter{Introduction}
The following report makes up one the two parts of a project in
computational group theory, the other part being a software package
for the computer system \GAP~(see \cite{GAP}), which can be found on
the WWW at the URL given on the title page. This text will describe
the mathematics that provide the foundations of the package, including
the algorithms used and their complexity, and also some of the more
computer science oriented aspects, like what datastructures that were
used, and how the implementation was done.

Computational group theory (CGT) is an area of research on the border
between group theory and computer science, and work in CGT is often of
both theoretical (mathematical) and practical (programming) nature,
leading to both theoretical results (mathematical theorems and proofs)
and practical results (software), and this project is no exception.
Introductory surveys of CGT can be found in \cite{sims98},
\cite{seress97}, \cite{neubuser95} and \cite{cannon92}.

The aim of the project was to make a \GAP~package with an
implementation of the Schreier-Sims algorithm for matrix groups. The
Schreier-Sims algorithm computes a base and a strong generating set
for a group, and an implementation of this fundamental algorithm is
already included in the standard \GAP~distribution, but that
implementation always first computes a faithful action (ie a
permutation representation) of the given group and then executes the
algorithm on the resulting permutation group. The idea for this
project was to restrict attention to matrix groups, and implement a
version of the algorithm which works with the matrices directly, and
see if one can obtain a more efficient implementation in this way.

A survey of computational matrix group theory can be found in
\cite{niemeyer01}. It should be noted that we are only interested in
finite groups, ie matrix groups over finite fields, and therefore
there is no need to worry about any noncomputability or undecidability
issues.

We will begin with a quick reference of the basic concepts from group
theory and computer science that are being used, before moving on to
describe the Schreier-Sims algorithm. The description will be quite
detailed, and then we will turn to the variants of the algorithm that
have also been implemented in the project: the random (ie.
probabilistic) Schreier-Sims algorithm and the
Schreier-Todd-Coxeter-Sims algorithm. After that we will say something
about the implementation, and describe some tricks and
improvements that have been done to make the algorithm faster.
Finally, the practical performance and benchmark results of the
implementation will be shown, and compared to the existing
implementation in \GAP.

It must be mentioned that a report similar to this one is
\cite{murray93}, from which a fair amount of inspiration comes.
