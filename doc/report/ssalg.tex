\chapter{The Schreier-Sims algorithm}

We will now describe the Schreier-Sims algorithm, and the references we are using are \cite{soicher98}, \cite{seress03} and \cite{butler91}. First we have to describe some background and clarify the problem that the algorithm solves.

\section{Background and motivation} \label{ss_background}
The overall goal for work in group theory is to understand groups, and
to answer various questions about groups. In particular, in CGT the
interest is focused on questions of algorithmic nature, so given a
group $G$ we are interested in things like
\begin{itemize}
\item What is $\abs{G}$?
\item We want to list all elements of $G$, without repetition.
\item If $G \leq H$ and we are given an arbitrary $g \in H$, is it true that $g \in G$? This is referred to as \emph{the membership problem}.
\item We want to generate a random $g \in G$.
\item Is $G$ abelian (soluble, polycyclic, nilpotent)?
\item Find representatives for the conjugacy classes of $G$.
\item Find a composition series for $G$, including the (isomorphism classes of the) composition factors.
\item Given $g \in G$ or $H \leq G$ find the centralizer of $g$ or the normalizer of $H$, respectively.
\end{itemize}
To accomplish these tasks we need a computer representation of $G$, ie
a datastructure. A common way in which a group is given is via a
generating set, but this alone does not help in solving our problems,
so we need a better representation. It must be possible to compute
this representation from a generating set, and a nice property would
be if subgroups of $G$ in some direct way inherit the representation,
so that divide-and-conquer techniques could be used when designing
algorithms (see \cite{clr90}).

\section{Base and strong generating set}
Consider the situation where we have a chain of subgroups of $G$
\begin{equation}
G = G^0 \geq G^1 \geq \dotsb \geq G^n = 1
\end{equation}
Each $g \in G$ can be written $g = u_1 g_1$ where $u_1$ is a
representative of $gG^1$ and $g_1 \in G^1$, and inductively we can
factorize $g$ as $g = u_1 u_2 \dotsm u_n$, since the subgroup chain
reaches $1$. Moreover, this factorization is \emph{unique} since for each $i$ the cosets of $G^{i + 1}$ partition $G^i$, and by the same reason, different group elements will have different factorizations.

We thus see that if we know generating sets for the subgroups in such
a subgroup chain, and if we know a right transversal $T^i$ of the cosets of
$G^{i + 1}$ in $G^i$ for each $i = 0, \dotsc, n - 1$, then we could easily
solve at least the first two listed problems listed in section
\ref{ss_background}. By Lagrange we know $\abs{G} = \abs{T^1}
\abs{G^1}$ and inductively $\abs{G} = \abs{T^1} \dotsm \abs{T^n}$,
which solves the first problem. Using the factorization we have a
bijection from $G$ to $T^1 \times T^2 \times \dotsb \times T^n$ so by
enumerating elements of the latter set and multiplying in $G$ we can list the elements of $G$ without repetition.

Now we introduce a special type of subgroup chain.
\begin{deff}
  Let $G$ be a group acting on the finite set $X$. A sequence of points
  $(\alpha_1, \dotsc, \alpha_n)$ of $X$ such that
  $G_{\alpha_1, \dotsc, \alpha_n} = 1$ is called a \emph{base} for
  $G$. Note that the base determines a \emph{stabiliser chain}
\begin{equation}
G \geq G_{\alpha_1} \geq \dotsb \geq G_{\alpha_1, \dotsc, \alpha_n} = 1
\end{equation}
Let $G^i = G_{\alpha_1, \dotsc, \alpha_i}$ for $i = 1, \dotsc, n$. A generating set $S$ for $G$ such that $\gen{S \cap G^i} = G^i$ for all $i$ is called a \emph{strong generating set} (SGS) for $G$.
\end{deff}

The concept of a base and strong generating was first introduced in
\cite{sims70} in the context of permutation groups, and is of
fundamental importance, though mainly for permutation groups. We
already know that the first two of our problems can be solved if we
know a subgroup chain and we shall see that with a base and strong
generating set the membership problem can also easily be solved. The
Schreier-Sims algorithm computes a base and strong generating set for
a group $G$ given a generating set $S$, and since it is an efficient
algorithm if $G$ is a permutation group, the concept of base and
strong generating set has become very important. Many more
sophisticated algorithms for permutation groups require a base and
strong generating set as input. For matrix groups, on the other hand,
the situation is a bit more complicated, as we shall see later.

\section{Schreier trees}
Before giving the Schreier-Sims algorithm itself, there are a few
auxiliary algorithms that must be explained. Consider therefore a
group $G = \gen{S}$ that acts on a finite set $X$. From the previous
section we know that even if we have a base $(\alpha_1, \dotsc,
\alpha_n)$ for $G$, with corresponding stabiliser chain $G \geq G^1
\geq \dotsb G^n$, we also need to find right transversals of the cosets of
$G^{i + 1}$ in $G^i$ for $i = 1, \dotsc, n - 1$. But since these groups
are stabilisers from the action on $X$, we can use
\ref{thm_orbit_stab} and instead find the orbits $\alpha_1^G,
\alpha_2^{G^1}, \dotsc, \alpha_n^{G^{n - 1}}$. We will see that the orbits are straightforward to compute.

The action of $G$ on $X$ can be represented by a labelled graph with
labels from $S$, analogous to the Cayley graph $\C_S(G)$ of $G$ from
\ref{def_cayley}. Let the vertices of the graph be $X$ and the edges
be $\{(p, p^g) \mid p \in X, g \in G\}$, where the edge $(p, p^g)$ is
labelled by $g$. Obviously the orbits of the action are the connected
components of this graph, so $\alpha_1^G$ is the component containing
$\alpha_1$. We are interested in finding this orbit and to store it in
a datatstructure, and we are thus led to consider a spanning tree of
the connected component, since such a tree contains enough information
for us. The edges of the graph that are left out do not give us any
additional relevant information about the action of $G$, only
alternative ways to move between the points, and trees are
considerably easier to store than graphs.

\begin{deff} \label{def_bsgs}
  Let $G$ be a group acting on the finite set $X$ and let $\alpha \in
  X$. A spanning tree for the component in the corresponding graph
  containing $\alpha$, that is rooted at $\alpha$, is called a
  \emph{Schreier tree} for the orbit $\alpha^G$.
\end{deff}

The Schreier tree can be computed by a simple breadth-first search of
the component containing $\alpha_1$, and thus we have an algorithm for
finding the orbits. However, it is of course not a computationally
good idea to explictly generate the graph, then compute the conntected
components and finally find the Schreier trees. A breadth-first search
to find the Schreier tree can be done without the graph itself, as
demonstrated by \ref{alg:orbit}.

\begin{algorithm} 
\dontprintsemicolon
\caption{\texttt{ComputeSchreierTree}}
\SetKwFunction{Tree}{Tree}
\SetKwFunction{AddChild}{AddChild}
\SetKwData{Points}{points}
\SetKwData{STree}{tree}
\SetKwData{Children}{children}
\KwData{A group $G = \gen{S}$ acting on a finite set $X$ and a point $\alpha \in X$.}
\KwResult{A Schreier tree for $\alpha^G$.}
\tcc{Assumes the existence of a function \texttt{Tree}$(x)$ that creates a tree with root $x$ and a function \texttt{AddChild}($T$, $p_1$, $p_2$, $l$) that adds $p_2$ as a child to $p_1$ in the tree $T$, with label $l$.}
\Begin{
  $\Points := \set{\alpha}$ \;
  $\Children := \emptyset$ \;
  $\STree := \Tree(\alpha)$ \;
  \Repeat{$\Points = \emptyset$}{
    \ForEach{$p \in \Points$}{
      \ForEach{$s \in S$}{
        $p^{\prime} := p^s$ \;
        \If{$p^{\prime} \notin \STree$}{
          $\AddChild(\STree, p, p^{\prime}, s)$ \;
          $\Children := \Children \cup \set{p^{\prime}}$ \;
        }
      }
    }
    $\Points := \Children$ \;
  }
  \Return{$\STree$}
}
\refstepcounter{algorithm}
\label{alg:orbit}
\end{algorithm}
We noted above that we used \ref{thm_orbit_stab} to store the orbits
instead of the coset representatives, but we still need the latter, so
we must be able to compute them. That is straightforward to do when we
have a Schreier tree, however.  Assume we have a Schreier tree $T =
(V, E)$ for the orbit $\alpha^G$ for some point $\alpha \in X$. If $g
\in G$ then $\alpha^g \in V$ and there is a path from $\alpha$ to
$\alpha^g$ in $T$. Let $s_1, s_2, \dotsc, s_n$ be the labels of the
path, so that $s_i \in S$ for $i = 1, \dotsc, n$, and let $h = s_1 s_2
\dotsm s_n$. Then obviosly $\alpha^h = \alpha^g$ so $G_{\alpha} g = 
G_{\alpha} h$ and $h$ is a coset representative for the coset of $g$.
Moreover, $h$ is unique since $T$ is a tree. Thus, to find the coset
representative for $g$ we only have to follow the unique path from
$\alpha^g$ to the root $\alpha$ and multiply the edge labels. \ref{alg:trace} performs the slightly more general task of
following the path from a given point to the root.

\begin{algorithm} 
\dontprintsemicolon
\caption{\texttt{OrbitElement}}
\SetKwFunction{EdgeLabel}{EdgeLabel}
\SetKw{KwRet}{break}
\SetKwData{Edge}{edge}
\KwData{A group $G = \gen{S}$ acting on a finite set $X$, a Schreier tree $T$ for the orbit $\alpha^G$ of the point $\alpha \in X$ and an arbitrary point $p \in X$.}
\KwResult{The element $g \in G$ such that $\alpha^g = p$}
\tcc{Assumes the existence of a function \texttt{EdgeLabel}$(T, p)$ that returns the label of the unique edge between $p$ and its parent in $T$}
\Begin{
  $g := 1$ \;
  \While{$p \neq \alpha$}{
    $s := \EdgeLabel(T, p)$ \;
    $p := p^{s^{-1}}$ \;
    $g := s g$ \;
  }
  \Return{$g$}
}
\refstepcounter{algorithm}
\label{alg:trace}
\end{algorithm}

We defer the complexity analysis of these algorithms until later.

\section{Formulating the problem}
To more formally state the problem solved by the Schreier-Sims algorithm, the following is needed.
\begin{deff}
  Let $G$ be a group acting on the finite set $X$. A sequence of points
  $B = (\alpha_1, \dotsc, \alpha_n)$ of $X$ and a generating set $S$ for $G$, such that no element of $S$ fixes all points of $B$, is called a \emph{partial base} and \emph{partial strong generating set}, respectively. 
\end{deff}
\begin{remark}
A base and strong generating set as in \ref{def_bsgs} are called \emph{complete}.
\end{remark}
\begin{remark} \label{rmk_partial_bsgs}
  If we define $G^i = G_{\alpha_1, \dotsc, \alpha_i}$, $S^i = S \cap
  G^i$ and $H^i = \gen{S^i}$ for $i = 1, \dotsc, n$, we see that $H^n
  = 1$ since no element of $S$ fixes all points of $B$ (and we use the
  convention that $\gen{\emptyset} = 1$). We therefore have
\begin{align}
G & \geq G^1 \geq \dotsb \geq G^n \\
G & \geq H^1 \geq \dotsb \geq H^n = 1 
\end{align}
Moreover, $G^i \geq \gen{S^i} = H^i$ for $i = 1, \dotsc, n$ and if we
have equality then by \ref{def_bsgs}, $S$ and $B$ are complete. If $h
\in H^{i + 1} = \gen{S^{i + 1}} = \gen{S \cap G^{i + 1}}$ then $h = s_1 \dotsm s_k$ where $\alpha_{i + 1}^{s_j} = \alpha_{i + 1}$ for $j = 1, \dotsc, k$ so $\alpha_{i + 1}^h = \alpha_{i + 1}$ and therefore $h \in H^i_{\alpha_{i + 1}}$. Thus $H^{i + 1} \leq H^i_{\alpha_{i + 1}}$ for $i = 0, \dotsc, n - 1$.
\end{remark}

Now our problem can be stated as follows: given a group $G$ acting on the
finite set $X$, together with a partial base $B$ with points from $X$
and partial strong generating set $S$, either verify that $B$ is a
(complete) base and that $S$ is a (complete) strong generating set, or
extend $B$ and $S$ so that they become complete. This is the problem that is
solved by the Schreier-Sims algorithm.


 The following result from \cite{leon80} is used when designing the algorithm.
\begin{theorem}
  Let $G$ be a group acting on the finite set $X$, and let $B =
  (\alpha_1, \dotsc, \alpha_n)$ be a partial base and $S$ a partial
  strong generating set for $G$. Let also $G^i = G_{\alpha_1, \dotsc,
    \alpha_i}$, $S^i = S \cap G^i$, $H^i = \gen{S^i}$ for $i = 1,
  \dotsc, n$ and $G = G^0 = H^0$. Then the following statements are equivalent:
\begin{enumerate}
\item $B$ and $S$ are complete. \label{leon80_1}
\item $G^i = H^i$ for $i = 0, \dotsc, n$. \label{leon80_2}
\item $H^i_{\alpha_{i + 1}} = H^{i + 1}$ for $i = 0, \dotsc, n - 1$. \label{leon80_3}
\item $[H^i : H^{i + 1}] = \abs{\alpha_{i+1}^{H^i}}$ for $i = 0, \dotsc, n - 1$. \label{leon80_4}
\end{enumerate}
\end{theorem}
\begin{proof}
From \ref{rmk_partial_bsgs} we know that \eqref{leon80_1} and \eqref{leon80_2} are equivalent. Assuming \eqref{leon80_2} we have
\begin{equation}
H^i_{\alpha_{i + 1}} = G^i_{\alpha_{i + 1}} = G^{i + 1} = H^{i + 1}
\end{equation}
for $i = 0, \dotsc, n - 1$ which is precisely \eqref{leon80_3}. If we instead assume \eqref{leon80_3} and also assume for induction that $G^i = H^i$ (the basis case $G = H^0 = G^0$ is ok) then
\begin{equation}
G^{i + 1} = G^i_{\alpha_{i + 1}} = H^i_{\alpha_{i + 1}} = H^{i + 1}
\end{equation}
so by induction we get $G^i = H^i$ for $i = 0, \dotsc, n$, which is \eqref{leon80_2}.

Now assume \eqref{leon80_3} and note that from \ref{thm_orbit_stab} we
have $[H^i : H^i_{\alpha_{i + 1}}] = \abs{\alpha_{i+1}^{H^i}}$, so
since $H^i_{\alpha_{i + 1}} = H^{i + 1}$ we get \eqref{leon80_4}.
Finally, assume \eqref{leon80_4}. From \ref{rmk_partial_bsgs} we know
$H^i_{\alpha_{i + 1}} \geq H^{i + 1}$ so if we again use
\ref{thm_orbit_stab} we get $\abs{\alpha_{i+1}^{H^i}} = [H^i : H^{i +
  1}] \geq [H^i : H^i_{\alpha_{i + 1}}] = \abs{\alpha_{i+1}^{H^i}}$
and thus $H^i_{\alpha_{i + 1}} = H^{i + 1}$.
\end{proof}

As observed earlier, we are often given a group in the form of a generating set, but Schreier-Sims algorithm requires a partial base and a partial strong generating set as input. Those are easy to compute, though, using \ref{alg:partial_bsgs}. The algorithm also makes sure that the partial strong generating set is closed under inverses, which makes some implementation issues more uniform.

\begin{algorithm} 
\dontprintsemicolon
\caption{\texttt{GetPartialBSGS}}
\SetKwFunction{NewBasePoint}{NewBasePoint}
\SetKwData{Base}{base}
\SetKwData{SGS}{sgs}
\SetKwData{Point}{point}
\KwData{A group $G = \gen{S}$ acting on a finite set $X$ and a sequence of points $B$ of $X$ (possibly empty).}
\KwResult{A partial base $B^{\prime}$ and partial strong generating set $S^{\prime}$ for $G$.}
\tcc{Assumes the existence of a function \texttt{NewBasePoint}$(g)$ that returns a point $p \in X$ such that $p^g \neq p$}
\Begin{
  $\Base := B$ \;
  $\SGS := \emptyset$ \;
  \ForEach{$s \in S \setminus \set{1}$}{
    \If{$\Base^s = \Base$}{
      $\Point := \NewBasePoint(s)$ \;
      $\Base := \Base \cup \set{\Point}$ \;
    }
    $\SGS := \SGS \cup \set{s, s^{-1}}$ \;
  }    
  \Return{$(\Base, \SGS)$}
}
\refstepcounter{algorithm}
\label{alg:partial_bsgs}
\end{algorithm}

\section{Schreier's Lemma}
The name Schreier in Schreier-Sims algorithm comes from the following
result, which allows us to find a generating set for a stabiliser. It
first appeared in \cite{schreier27}, and our proof is originally from \cite{hall59}.
\begin{theorem}[Schreier's Lemma]
Let $G = \gen{S}$ be a group, let $H \leq G$ and let $T$ be a right transversal of the cosets of $H$ in $G$. For $g \in G$, let $\bar{g} \in T$ be the unique element such that $Hg = H\bar{g}$. Then $H$ is generated by
\begin{equation}
S_H = \set{t s (\overline{ts})^{-1} \mid t \in T, s \in S}
\end{equation}
\end{theorem}
\begin{proof}
  Without loss of generality we can assume that $1 \in T$ (the coset
  representative of $H$ itself). By definition, $Hts = H\overline{ts}$
  which implies that $t s (\overline{ts})^{-1} \in H$ for all $t \in
  T, s \in S$. Hence, $S_H \subseteq H$ and $\gen{S_H} \leq H$, so the
  content of the statement lies in the other inclusion.
  
  Let $h \in H \leq G$ and observe that since $\gen{S} = G$ we have $h
  = s_1 s_2 \dotsm s_k$ where $s_i \in S \cup S^{-1}$ for $i = 1,
  \dotsc, k$. Define a sequence $t_1, t_2, \dotsc, t_{k + 1}$ of $k +
  1$ elements of $T$ as follows: $t_1 = 1$ and inductively $t_{i + 1} = \overline{t_i s_i}$. Furthermore, let $a_i = t_i s_i t_{i + 1}^{-1}$ for $i = 1, \dotsc, n$ and observe that
\begin{equation}
h = (t_1 s_1 t_2^{-1})(t_2 s_2 t_3^{-1}) \dotsm (t_n s_n t_{n + 1}^{-1}) t_{n + 1} = a_1 a_2 \dotsm a_n t_{n + 1}
\end{equation}
We now show that $a_i \in \gen{S_H}$ for $i = 1, \dotsc, n$, and that $t_{n + 1} = 1$, which
implies that $H \leq \gen{S_H}$. 

For each $i = 1, \dotsc, n$, either $s_i \in S$ or $s_i^{-1} \in S$.
In the first case we immediately get $a_i = t_i s_i (\overline{t_i
  s_i})^{-1} \in S_H$, and in the second case we have $H t_{i + 1}
s_i^{-1} = H \overline{t_i s_i} s_i^{-1} = H t_i s_i s_i^{-1} = H t_i$
which implies that $t_i = \overline{t_{i + 1} s_i^{-1}}$. Hence,
$a_i^{-1} = t_{i + 1} s_i^{-1} t_i^{-1} = t_{i + 1} s_i^{-1}
\overline{t_{i + 1} s_i^{-1}}^{-1} \in S_H$ and thus $a_i \in \gen{S_H}$.

Finally, since $h \in H$ and $\gen{S_H} \leq H$ we have $t_{n + 1} = (a_1 a_2 \dotsm a_n)^{-1} h \in H$, so $t_{n + 1}$ is the coset representative of $H$, and therefore $t_{n + 1} = 1$. Thus $H \leq \gen{S_H}$.
\end{proof}

\subsection{Decreasing number of generators}

\subsection{Computing a base and SGS}

\section{The main algorithm}

\subsection{Membership testing}

\section{Complexity analysis}

\subsection{Complexity of Schreier tree algorithms}
For the analysis of \ref{alg:orbit} we assume that the external functions \texttt{Tree} and \texttt{AddChild}, that depend on the particular datastructure used, take $O(1)$ time. This is a reasonable assumption and it is satisfied in the code for the project. $G \leq GL(d, q)$, $O(\abs{X} \abs{S}d)$
