\chapter{The Schreier-Sims algorithm}

We will now describe the Schreier-Sims algorithm, and the references we are using are \cite{soicher98}, \cite{seress03} and \cite{butler91}. First we have to describe some background and clarify the problem that the algorithm solves.

\section{Background and motivation} \label{ss_background}
The overall goal for work in group theory is to understand groups, and
to answer various questions about groups. In particular, in CGT the
interest is focused on questions of algorithmic nature, so given a
group $G$ we are interested in algorithms for things like
\begin{itemize}
\item What is $\abs{G}$?
\item We want to list all elements of $G$, without repetition.
\item If $G \leq H$ and we are given an arbitrary $g \in H$, is it true that $g \in G$? This is referred to as \emph{the membership problem}.
\item We want to generate a random $g \in G$.
\item Is $G$ abelian (soluble, polycyclic, nilpotent)?
\item Find representatives for the conjugacy classes of $G$.
\item Find a composition series for $G$, including the (isomorphism classes of the) composition factors.
\item Given $g \in G$ or $H \leq G$ find the centralizer of $g$ or the normalizer of $H$, respectively.
\end{itemize}
To accomplish these tasks we need a computer representation of $G$, ie
a datastructure. A common way in which a group is given is via a
generating set, but this alone does not help in solving our problems,
so we need a better representation. It must be possible to compute
this representation from a generating set, and a nice property would
be if subgroups of $G$ in some direct way inherit the representation,
so that divide-and-conquer techniques could be used when designing
algorithms (see \cite{clr90}).

\section{Base and strong generating set} \label{section:bsgs}
Consider the situation where we have a chain of subgroups of $G$
\begin{equation}
G = G^0 \geq G^1 \geq \dotsb \geq G^n = 1
\end{equation}
Each $g \in G$ can be written $g = g_1 u_1$ where $u_1$ is a
representative of $G^1 g$ and $g_1 \in G^1$, and inductively we can
factorize $g$ as $g = u_n u_{n - 1} \dotsm u_1$, since the subgroup chain
reaches $1$. Moreover, this factorization is \emph{unique} since for each $i$ the cosets of $G^{i + 1}$ partition $G^i$, and by the same reason, different group elements will have different factorizations.

We thus see that if we know generating sets for the subgroups in such
a subgroup chain, and if we know a right transversal $T^i$ of the cosets of
$G^{i + 1}$ in $G^i$ for each $i = 0, \dotsc, n - 1$, then we could easily
solve at least the first two listed problems listed in section
\ref{ss_background}. By Lagrange we know $\abs{G} = \abs{T^1}
\abs{G^1}$ and inductively $\abs{G} = \abs{T^1} \dotsm \abs{T^n}$,
which solves the first problem. Using the factorization we have a
bijection from $G$ to $T^1 \times T^2 \times \dotsb \times T^n$ so by
enumerating elements of the latter set and multiplying in $G$ we can list the elements of $G$ without repetition.

Now we introduce a special type of subgroup chain.
\begin{deff}
  Let $G$ be a group acting on the finite set $X$. A sequence of points
  $(\alpha_1, \dotsc, \alpha_n)$ of $X$ such that
  $G_{\alpha_1, \dotsc, \alpha_n} = 1$ is called a \emph{base} for
  $G$. Note that the base determines a \emph{stabiliser chain}
\begin{equation}
G \geq G_{\alpha_1} \geq \dotsb \geq G_{\alpha_1, \dotsc, \alpha_n} = 1
\end{equation}
Let $G^i = G_{\alpha_1, \dotsc, \alpha_i}$ for $i = 1, \dotsc, n$. A generating set $S$ for $G$ such that $\gen{S \cap G^i} = G^i$ for all $i$ is called a \emph{strong generating set} (SGS) for $G$.
\end{deff}

The concept of a base and strong generating was first introduced in
\cite{sims70} in the context of permutation groups, and is of
fundamental importance, though mainly for permutation groups. We
already know that the first two of our problems can be solved if we
know a subgroup chain and we shall see that with a base and strong
generating set the membership problem can also easily be solved. The
Schreier-Sims algorithm computes a base and strong generating set for
a group $G$ given a generating set $S$, and since it is an efficient
algorithm if $G$ is a permutation group, the concept of base and
strong generating set has become very important. Many more
sophisticated algorithms for permutation groups require a base and
strong generating set as input. For matrix groups, on the other hand,
the situation is a bit more complicated, as we shall see later.

\section{Schreier trees}
Before giving the Schreier-Sims algorithm itself, there are a few
auxiliary algorithms that must be explained. Consider therefore a
group $G = \gen{S}$ that acts on a finite set $X$. From the previous
section we know that even if we have a base $(\alpha_1, \dotsc,
\alpha_n)$ for $G$, with corresponding stabiliser chain $G \geq G^1
\geq \dotsb G^n = 1$, we also need to find right transversals of the cosets of
$G^{i + 1}$ in $G^i$ for $i = 1, \dotsc, n - 1$. But since these groups
are stabilisers from the action on $X$, we can use
\ref{thm_orbit_stab} and instead find the orbits $\alpha_1^G,
\alpha_2^{G^1}, \dotsc, \alpha_n^{G^{n - 1}}$. We will see that the orbits are straightforward to compute.

The action of $G$ on $X$ can be represented by a labelled graph with
labels from $S$, analogous to the Cayley graph $\C_S(G)$ of $G$ from
\ref{def_cayley}. Let the vertices of the graph be $X$ and the edges
be $\{(p, p^g) \mid p \in X, g \in G\}$, where the edge $(p, p^g)$ is
labelled by $g$. Obviously the orbits of the action are the connected
components of this graph, so $\alpha_1^G$ is the component containing
$\alpha_1$. We are interested in finding this orbit and to store it in
a datatstructure, and we are thus led to consider a spanning tree of
the connected component, since such a tree contains enough information
for us. The edges of the graph that are left out do not give us any
additional relevant information about the action of $G$, only
alternative ways to move between the points, and trees are
considerably easier to store than graphs.

\begin{deff} \label{def_bsgs}
  Let $G$ be a group acting on the finite set $X$ and let $\alpha \in
  X$. A spanning tree for the component in the corresponding graph
  containing $\alpha$, that is rooted at $\alpha$, is called a
  \emph{Schreier tree} for the orbit $\alpha^G$.
\end{deff}

The Schreier tree can be computed by a simple breadth-first search of
the component containing $\alpha_1$, and thus we have an algorithm for
finding the orbits. However, it is of course not a computationally
good idea to explictly generate the graph, then compute the conntected
components and finally find the Schreier trees. A breadth-first search
to find the Schreier tree can be done without the graph itself, as
demonstrated by \ref{alg:orbit}.

\begin{algorithm} 
\dontprintsemicolon
\caption{\texttt{ComputeSchreierTree}}
\SetKwFunction{Tree}{Tree}
\SetKwFunction{AddChild}{AddChild}
\SetKwData{Points}{points}
\SetKwData{STree}{tree}
\SetKwData{Children}{children}
\KwData{A group $G = \gen{S}$ acting on a finite set $X$ and a point $\alpha \in X$.}
\KwResult{A Schreier tree for $\alpha^G$.}
\tcc{Assumes the existence of a function \texttt{Tree}$(x)$ that creates an empty tree with root $x$ and a function \texttt{AddChild}($T$, $p_1$, $p_2$, $l$) that adds $p_2$ as a child to $p_1$ in the tree $T$, with label $l$.}
\Begin{
  $\Points := \set{\alpha}$ \;
  $\STree := \Tree(\alpha)$ \;
  \Repeat{$\Points = \emptyset$}
  {
    $\Children := \emptyset$ \;
    \ForEach{$p \in \Points$}
    {
      \ForEach{$s \in S$}
      {
        $p^{\prime} := p^s$ \; \label{alg:orbit_innerloop1}
        \If{$p^{\prime} \notin \STree$}
        {
          $\AddChild(\STree, p, p^{\prime}, s)$ \;
          $\Children := \Children \cup \set{p^{\prime}}$ \; \label{alg:orbit_innerloop2}
        } 
      }
    }
    $\Points := \Children$ \;
  }
  \Return{$\STree$}
}
\refstepcounter{algorithm}
\label{alg:orbit}
\end{algorithm}
We noted above that we used \ref{thm_orbit_stab} to store the orbits
instead of the coset representatives, but we still need the latter, so
we must be able to compute them. That is straightforward to do when we
have a Schreier tree, however.  Assume we have a Schreier tree $T =
(V, E)$ for the orbit $\alpha^G$ for some point $\alpha \in X$. If $g
\in G$ then $\alpha^g \in V$ and there is a path from $\alpha$ to
$\alpha^g$ in $T$. Let $s_1, s_2, \dotsc, s_n$ be the labels of the
path, so that $s_i \in S$ for $i = 1, \dotsc, n$, and let $h = s_1 s_2
\dotsm s_n$. Then obviously $\alpha^h = \alpha^g$ so $G_{\alpha} g = 
G_{\alpha} h$ and $h$ is a coset representative for the coset of $g$.
Moreover, $h$ is unique since $T$ is a tree. Thus, to find the coset
representative for $g$ we only have to follow the unique path from
$\alpha^g$ to the root $\alpha$ and multiply the edge labels. \ref{alg:trace} performs the slightly more general task of
following the path from a given point to the root.

\begin{algorithm} 
\dontprintsemicolon
\caption{\texttt{OrbitElement}}
\SetKwFunction{EdgeLabel}{EdgeLabel}
\SetKw{KwRet}{break}
\SetKwData{Edge}{edge}
\KwData{A group $G = \gen{S}$ acting on a finite set $X$, a Schreier tree $T$ for the orbit $\alpha^G$ of the point $\alpha \in X$ and an arbitrary point $p \in X$.}
\KwResult{The element $g \in G$ such that $\alpha^g = p$}
\tcc{Assumes the existence of a function \texttt{EdgeLabel}$(T, p)$ that returns the label of the unique edge between $p$ and its parent in $T$}
\Begin
{
  $g := 1$ \;
  \While{$p \neq \alpha$}
  {
    $s := \EdgeLabel(T, p)$ \;
    $p := p^{s^{-1}}$ \;
    $g := s g$ \;
  }
  \Return{$g$}
}
\refstepcounter{algorithm}
\label{alg:trace}
\end{algorithm}

We defer the complexity analysis of these algorithms until later.

\section{Formulating the problem}
To more formally state the problem solved by the Schreier-Sims algorithm, the following is needed.
\begin{deff}
  Let $G$ be a group acting on the finite set $X$. A sequence of points
  $B = (\alpha_1, \dotsc, \alpha_n)$ of $X$ and a generating set $S$ for $G$, such that no element of $S$ fixes all points of $B$, is called a \emph{partial base} and \emph{partial strong generating set}, respectively. 
\end{deff}
\begin{remark}
A base and strong generating set as in \ref{def_bsgs} are called \emph{complete}.
\end{remark}
\begin{remark} \label{rmk_partial_bsgs}
  If we define $G^i = G_{\alpha_1, \dotsc, \alpha_i}$, $S^i = S \cap
  G^i$ and $H^i = \gen{S^i}$ for $i = 1, \dotsc, n$, we see that $H^n
  = 1$ since no element of $S$ fixes all points of $B$ (and we use the
  convention that $\gen{\emptyset} = 1$). We therefore have
\begin{align}
G & \geq G^1 \geq \dotsb \geq G^n \\
G & \geq H^1 \geq \dotsb \geq H^n = 1 
\end{align}
Moreover, $G^i \geq \gen{S^i} = H^i$ for $i = 1, \dotsc, n$ and if we
have equality then, by \ref{def_bsgs}, $S$ and $B$ are complete. If $h
\in H^{i + 1} = \gen{S^{i + 1}} = \gen{S \cap G^{i + 1}}$ then $h = s_1 \dotsm s_k$ where $\alpha_{i + 1}^{s_j} = \alpha_{i + 1}$ for $j = 1, \dotsc, k$ so $\alpha_{i + 1}^h = \alpha_{i + 1}$ and therefore $h \in H^i_{\alpha_{i + 1}}$. Thus $H^{i + 1} \leq H^i_{\alpha_{i + 1}}$ for $i = 0, \dotsc, n - 1$.
\end{remark}

Now our problem can be stated as follows: given a group $G$ acting on the
finite set $X$, together with a partial base $B$ with points from $X$
and partial strong generating set $S$, either verify that $B$ is a
(complete) base and that $S$ is a (complete) strong generating set, or
extend $B$ and $S$ so that they become complete. This is the problem that is
solved by the Schreier-Sims algorithm.


 The following result from \cite{leon80} is used when designing the algorithm.
\begin{theorem} \label{thm_leon}
  Let $G$ be a group acting on the finite set $X$, and let $B =
  (\alpha_1, \dotsc, \alpha_n)$ be a partial base and $S$ a partial
  strong generating set for $G$. Let also $G^i = G_{\alpha_1, \dotsc,
    \alpha_i}$, $S^i = S \cap G^i$, $H^i = \gen{S^i}$ for $i = 1,
  \dotsc, n$ and $G = G^0 = H^0$. Then the following statements are equivalent:
\begin{enumerate}
\item $B$ and $S$ are complete. \label{leon80_1}
\item $G^i = H^i$ for $i = 0, \dotsc, n$. \label{leon80_2}
\item $H^i_{\alpha_{i + 1}} = H^{i + 1}$ for $i = 0, \dotsc, n - 1$. \label{leon80_3}
\item $[H^i : H^{i + 1}] = \abs{\alpha_{i+1}^{H^i}}$ for $i = 0, \dotsc, n - 1$. \label{leon80_4}
\end{enumerate}
\end{theorem}
\begin{proof}
From \ref{rmk_partial_bsgs} we know that \eqref{leon80_1} and \eqref{leon80_2} are equivalent. Assuming \eqref{leon80_2} we have
\begin{equation}
H^i_{\alpha_{i + 1}} = G^i_{\alpha_{i + 1}} = G^{i + 1} = H^{i + 1}
\end{equation}
for $i = 0, \dotsc, n - 1$ which is precisely \eqref{leon80_3}. If we instead assume \eqref{leon80_3} and also assume for induction that $G^i = H^i$ (the basis case $G = H^0 = G^0$ is ok) then
\begin{equation}
G^{i + 1} = G^i_{\alpha_{i + 1}} = H^i_{\alpha_{i + 1}} = H^{i + 1}
\end{equation}
so by induction we get $G^i = H^i$ for $i = 0, \dotsc, n$, which is \eqref{leon80_2}.

Now assume \eqref{leon80_3} and note that from \ref{thm_orbit_stab} we
have $[H^i : H^i_{\alpha_{i + 1}}] = \abs{\alpha_{i+1}^{H^i}}$, so
since $H^i_{\alpha_{i + 1}} = H^{i + 1}$ we get \eqref{leon80_4}.
Finally, assume \eqref{leon80_4}. From \ref{rmk_partial_bsgs} we know
$H^i_{\alpha_{i + 1}} \geq H^{i + 1}$ so if we again use
\ref{thm_orbit_stab} we get $\abs{\alpha_{i+1}^{H^i}} = [H^i : H^{i +
  1}] \geq [H^i : H^i_{\alpha_{i + 1}}] = \abs{\alpha_{i+1}^{H^i}}$
and thus $H^i_{\alpha_{i + 1}} = H^{i + 1}$.
\end{proof}

As observed earlier, we are often given a group in the form of a
generating set, but Schreier-Sims algorithm requires a partial base
and a partial strong generating set as input. Those are easy to
compute, though, using \ref{alg:partial_bsgs}. The algorithm also
makes sure that the partial strong generating set is closed under
inverses, which makes some implementation issues more uniform. Later we will see how the function \texttt{NewBasePoint} that is used in \ref{alg:partial_bsgs} can be implemented when $G$ is a matrix group.

\begin{algorithm} 
\dontprintsemicolon
\caption{\texttt{GetPartialBSGS}}
\SetKwFunction{NewBasePoint}{NewBasePoint}
\SetKwData{Base}{base}
\SetKwData{SGS}{sgs}
\SetKwData{Point}{point}
\KwData{A group $G = \gen{S}$ acting on a finite set $X$ and a sequence of points $B$ of $X$ (possibly empty).}
\KwResult{A partial base $B^{\prime}$ and partial strong generating set $S^{\prime}$ for $G$.}
\tcc{Assumes the existence of a function \texttt{NewBasePoint}$(g)$ that returns a point $p \in X$ such that $p^g \neq p$}
\Begin
{
  $\Base := B$ \;
  $\SGS := \emptyset$ \;
  \ForEach{$s \in S \setminus \set{1}$}
  {
    \If{$\Base^s = \Base$} 
    { \label{alg:partial_bsgs_innerloop1}
      $\Point := \NewBasePoint(s)$ \;
      $\Base := \Base \cup \set{\Point}$ \; \label{alg:partial_bsgs_innerloop2} 
    }
    $\SGS := \SGS \cup \set{s, s^{-1}}$ \;
  }    
  \Return{$(\Base, \SGS)$}
}
\refstepcounter{algorithm}
\label{alg:partial_bsgs}
\end{algorithm}

\section{Schreier's Lemma}
The name Schreier in Schreier-Sims algorithm comes from the following
result, which in our case allows us to find a generating set for a stabiliser. It
first appeared in \cite{schreier27}, and our proof is originally from \cite{hall59}.
\begin{theorem}[Schreier's Lemma] \label{schreiers_lemma}
Let $G = \gen{S}$ be a group, let $H \leq G$ and let $T$ be a right transversal of the cosets of $H$ in $G$. For $g \in G$, let $\bar{g} \in T$ be the unique element such that $Hg = H\bar{g}$. Then $H$ is generated by
\begin{equation}
S_H = \set{t s (\overline{ts})^{-1} \mid t \in T, s \in S}
\end{equation}
\end{theorem}
\begin{proof}
  Without loss of generality we can assume that $1 \in T$ (the coset
  representative of $H$ itself). By definition, $Hts = H\overline{ts}$
  which implies that $t s (\overline{ts})^{-1} \in H$ for all $t \in
  T, s \in S$. Hence, $S_H \subseteq H$ and $\gen{S_H} \leq H$, so the
  content of the statement lies in the other inclusion.
  
  Let $h \in H \leq G$ and observe that since $\gen{S} = G$ we have $h
  = s_1 s_2 \dotsm s_k$ where $s_i \in S \cup S^{-1}$ for $i = 1,
  \dotsc, k$. Define a sequence $t_1, t_2, \dotsc, t_{k + 1}$ of $k +
  1$ elements of $T$ as follows: $t_1 = 1$ and inductively $t_{i + 1} = \overline{t_i s_i}$. Furthermore, let $a_i = t_i s_i t_{i + 1}^{-1}$ for $i = 1, \dotsc, n$ and observe that
\begin{equation}
h = (t_1 s_1 t_2^{-1})(t_2 s_2 t_3^{-1}) \dotsm (t_n s_n t_{n + 1}^{-1}) t_{n + 1} = a_1 a_2 \dotsm a_n t_{n + 1}
\end{equation}
We now show that $a_i \in \gen{S_H}$ for $i = 1, \dotsc, n$, and that $t_{n + 1} = 1$, which
implies that $H \leq \gen{S_H}$. 

For each $i = 1, \dotsc, n$, either $s_i \in S$ or $s_i^{-1} \in S$.
In the first case we immediately get $a_i = t_i s_i (\overline{t_i
  s_i})^{-1} \in S_H$, and in the second case we have $H t_{i + 1}
s_i^{-1} = H \overline{t_i s_i} s_i^{-1} = H t_i s_i s_i^{-1} = H t_i$
which implies that $t_i = \overline{t_{i + 1} s_i^{-1}}$. Hence,
$a_i^{-1} = t_{i + 1} s_i^{-1} t_i^{-1} = t_{i + 1} s_i^{-1}
\left(\overline{t_{i + 1} s_i^{-1}}\right)^{-1} \in S_H$ and thus $a_i \in \gen{S_H}$.

Finally, since $h \in H$ and $\gen{S_H} \leq H$ we have $t_{n + 1} = (a_1 a_2 \dotsm a_n)^{-1} h \in H$, so $t_{n + 1}$ is the coset representative of $H$, and therefore $t_{n + 1} = 1$. Thus $H \leq \gen{S_H}$.
\end{proof}

Our situtation is that we have a group $G = \gen{S}$ acting on the finite set
$X$, and we want to use Schreier's Lemma to find the generators
(usually called the \emph{Schreier generators}) for the stabiliser
$G_{\alpha}$, where $\alpha \in X$. If we compute a Schreier tree for
the orbit $\alpha^G$ using \ref{alg:orbit} then we know that
\ref{alg:trace} can be used to find the transversal of the cosets of
$G_{\alpha}$ in $G$. 

For $p \in X$, let $t(p) \in G$ denote the result of \ref{alg:trace}.
Using the notation in \ref{schreiers_lemma} we then have $\bar{g} =
t(\alpha^g)$ and the transversal is $\set{t(p) \mid p \in \alpha^G}$,
so for $s \in S$ and $p \in \alpha^G$ the Schreier generator can be
expressed as
\begin{equation} 
t(p) s t(\alpha^{t(p) s})^{-1} = t(p) s t((\alpha^{t(p)})^s)^{-1} = t(p) s t(p^s)^{-1}
\end{equation}
and a generating set for $G_{\alpha}$ is
\begin{equation} \label{gen_stabiliser}
\set{t(p) s t(p^s)^{-1} \mid p \in \alpha^G, s \in S}
\end{equation}

\subsection{Computing a base and SGS}
Going back to our problem, if we have a partial base $B = (\alpha_1,
\dotsc, \alpha_n)$ for $G$ with corresponding partial strong
generating set $S$, then we can use Schreier's Lemma to solve our
problem. Using the notation from Theorem \ref{thm_leon}, we calculate
the Schreier generators for each $G^i$, using \eqref{gen_stabiliser} and add them to $S$, possibly
adding points to $B$, if some Schreier generator fixes the whole base,
in order to ensure that $B$ and $S$ are still partial. When this is
finished, we have $H^i = G^i$ for $i = 1, \dotsc, n$ and thus $B$ and
$S$ are complete. We can use \ref{alg:get_schreier_gen} to calculate a Schreier generator.

\begin{algorithm} 
\dontprintsemicolon
\caption{\texttt{GetSchreierGenerator}}
\SetKwFunction{OrbitElement}{OrbitElement}
\KwData{A group $G = \gen{S}$ acting on a finite set $X$, a Schreier tree $T$ for the orbit $\alpha^G$ of the point $\alpha \in X$, a $p \in X$ and a generator $s \in S$.}
\KwResult{The Schreier generator corresponding to $p$ and $s$}
\Begin
{
  $t_1 := \OrbitElement(T, p)$ \;
  $t_2 := \OrbitElement(T, p^s)$ \;
  \Return{$t_1 s t_2$}
}
\refstepcounter{algorithm}
\label{alg:get_schreier_gen}
\end{algorithm}

However, there is a problem with this simple approach, in that the
generating sets defined in \eqref{gen_stabiliser} can be very large,
and contain many redundant generators. A fraction of the Schreier
generators is usually enough to generate the stabiliser. In
\cite{hall59} it is shown that $[G : G_{\alpha}] - 1$ of the Schreier
generators for $G_{\alpha}$ are equal to the identity. For instance,
if for some point $p \in \alpha^G$ we have that $(p, p^s)$ is an edge
of the Schreier tree for $\alpha^G$, then the Schreier generator $t(p)
s t(p^s)^{-1}$ is the identity. Thus, the number of non-trivial
Schreier generators can be as large as $(\abs{S} - 1) [G : G_{\alpha}]
+ 1$, though some of them may be equal to each other.

In our case we calculate the Schreier generators for each $G^i$, using $S^{i - 1}$ in place of $S$, and since $S^{i - 1}$ are precisely the calculated Schreier generators for $G^{i - 1}$, the number of non-trivial Schreier generators for $G^i$ may be as large as 
\begin{equation} \label{num_schreier_gens}
1 + (\abs{S} - 1) \prod_{j = 0}^i \abs{\alpha_{j + 1}^{G^j}}
\end{equation}
Since the orbit sizes are only bounded by $\abs{X}$, we see that
\eqref{num_schreier_gens} is exponential in $\abs{X}$, and therefore
this method may not be efficient.

\subsection{Reducing the number of generators}
It is possible to reduce the number of generators at each step, so
that our generating set never grows too large. This can be done using
\ref{alg:reduce_schreier_gens}, which is due to Sims, and which can also be
found in \cite{sims98} and \cite{soicher98}.

\begin{algorithm} 
\dontprintsemicolon
\caption{\texttt{BoilSchreierGenerators}}
\KwData{A group $G$ acting on a finite set $X$, a partial base $B = (\alpha_1, \dotsc, \alpha_k)$ and corresponding partial strong generating set $S$ for $G$, an integer $1 \leq m \leq k$.}
\KwResult{A smaller partial strong generating set for $G$}
\Begin
{
  \For{$i := 1$ \KwTo $m$}
  {
    $T := S^{i - 1}$ \;
    \ForEach{$g \in T$}
    {
      \ForEach{$h \in T$}
      {
        \If{$\alpha_i^g = \alpha_i^h \neq \alpha_i$}
        {
          $S := (S \setminus \set{h}) \cup \set{g h^{-1}}$ \;
        }
      }
    }    
  }
  \Return{$S$}
}
\refstepcounter{algorithm}
\label{alg:reduce_schreier_gens}
\end{algorithm}

This algorithm reduces the generating set to size $\binom{\abs{X}}{2}
\in \OR{\abs{X}^2}$. With this algorithm, we can solve our problem using
\ref{alg:call_ss} and \ref{alg:pre_ss}. This, however, is not the
Schreier-Sims algorithm, which is a more clever and efficient method
of performing the same things.

\begin{algorithm} 
\dontprintsemicolon
\caption{\texttt{ComputeBSGS}}
\SetKwFunction{GetPartialBSGS}{GetPartialBSGS}
\SetKwFunction{Schreier}{Schreier}
\SetKwFunction{BoilSchreierGenerators}{BoilSchreierGenerators}
\SetKwData{SGS}{sgs}
\SetKwData{Base}{base}
\KwData{A group $G = \gen{S}$ acting on a finite set $X$.}
\KwResult{A base and strong generating set for $G$}.
\Begin
{
  $(\Base, \SGS) := \GetPartialBSGS(S, \emptyset)$ \;
  \For{$i := 1$ \KwTo $\abs{\Base}$}
  {
    $(\Base, \SGS) := \Schreier(\Base, \SGS, i)$ \;
    $\SGS := \BoilSchreierGenerators(\Base, \SGS, i)$ \;
  }
  \Return{$(\Base, \SGS)$}
}
\refstepcounter{algorithm}
\label{alg:call_ss}
\end{algorithm}

\begin{algorithm} 
\dontprintsemicolon
\caption{\texttt{Schreier}}
\SetKwFunction{GetSchreierGenerator}{GetSchreierGenerator}
\SetKwFunction{ComputeSchreierTree}{ComputeSchreierTree}
\SetKwFunction{NewBasePoint}{NewBasePoint}
\SetKwData{Tree}{tree}
\SetKwData{Gen}{gen}
\SetKwData{Point}{point}
\KwData{A group $G$ acting on a finite set $X$, a partial base $B = (\alpha_1, \dotsc, \alpha_k)$ and corresponding partial strong generating set $S$ for $G$, an integer $1 \leq i \leq k$ such that $G^j = H^j$ for $j = 0, \dotsc, i - 1$.}
\KwResult{Possibly extended partial base $B = (\alpha_1, \dotsc, \alpha_m)$ and corresponding partial strong generating $S$ set for $G$ such that $G^j = H^j$ for $j = 0, \dotsc, i$.}
\tcc{Assumes the existence of a function \texttt{NewBasePoint}$(g)$ that returns a point $p \in X$ such that $p^g \neq p$}
\Begin
{
  $T := S^{i - 1}$ \;
  $\Tree := \ComputeSchreierTree(T, \alpha_i)$ \;
  \ForEach{$p \in \alpha_i^{H^{i - 1}}$}
  {
    \ForEach{$s \in T$}
    {
      $\Gen := \GetSchreierGenerator(\Tree, p, s)$ \;
      \If{$\Gen \neq 1$}
      {
        $S := S \cup \set{\Gen, \Gen^{-1}}$ \;
        \If{$B^{\Gen} = B$}
        {
          $\Point := \NewBasePoint(\Gen)$ \;
          $B := B \cup \set{\Point}$ \;
        }
      }
    }    
  }
  \Return{$(B, S)$}
}
\refstepcounter{algorithm}
\label{alg:pre_ss}
\end{algorithm}


\section{Membership testing}
We now assume that we know a (complete) base $B = (\alpha_1, \dotsc,
\alpha_n)$ and a (complete) strong generating set $S$ for our group
$G$, and we present an efficient algorithm for determining if, given
an arbitrary group element $g$, it is true that $g \in G$. The
implicit assumption is of course that $G \leq H$ for some large group
$H$ and that $g \in H$, but since we are interested in matrix groups
this is always true with $H$ being some general linear group.

This algorithm is used in the Schreier-Sims algorithm together with
Theorem \ref{thm_leon}, as we shall see later. 

Recall that if we have a base then there is an associated stabiliser
chain, and as described in section \ref{section:bsgs}, if $g \in G$ we
can factorise $g$ as a product of coset representatives $g = u_n u_{n - 1}
\dotsm u_1$ where $u_i$ is the representative of $G^i g$ in $G^{i -
  1}$. Moreover, if we for each $i = 1, \dotsc, n$ have computed a
Schreier tree $T_i$ for $\alpha_i^{G^{i - 1}}$ then we can use
\ref{alg:trace} to compute each coset representative.

More specifically, if $g \in G$ then $g = g_1 t(\alpha_1^g)$ where, as
before, $t(p)$ is the output of \ref{alg:trace} on the point $p$ and
$g_1 \in G^1$. On the other hand, if $g \notin G$ then either
$\alpha_1^g \notin \alpha^G$ or $g_1 \notin G_1$. To test if $g \in G$
we can therefore proceed inductively, and first check if $\alpha_1^g
\in \alpha^G$ and if that is true then test whether $g_1 \in G_1$.
This is formalised in \ref{alg:membership}.

\begin{algorithm} 
\dontprintsemicolon
\caption{\texttt{Membership}}
\SetKwFunction{OrbitElement}{OrbitElement}
\SetKwData{Tree}{tree}
\SetKwData{Element}{element}
\SetKwData{Point}{point}
\KwData{A group $G$ acting on a finite set $X$, a base $B = (\alpha_1, \dotsc, \alpha_n)$, a Schreier tree $T_i$ for the orbit $\alpha_{i + 1}^{G^i}$ for each $i = 0, \dotsc, n - 1$, and a group element $g$.}
\KwResult{A residue $r$ and drop-out level $1 \leq l \leq n + 1$.}
\Begin
{
  $r := g$ \;
  \For{$i := 1$ \KwTo $n$}
  {
    \If{$\alpha_i^r \notin T_{i - 1}$}
    {
      \Return{$(r, i)$}
    }
    $\Element := \OrbitElement(T_{i - 1}, \alpha_i^r)$ \;
    $r := r \cdot \Element^{-1}$ \;
  }
  \Return{$(r, n + 1)$} \label{alg:membership_return}
}
\refstepcounter{algorithm}
\label{alg:membership}
\end{algorithm}

The terminology \emph{residue} and \emph{level} is introduced here,
with obvious meanings. As can be seen, the algorithm returns the level
at which it fails, as this is needed in the Schreier-Sims algorithm. Note that even if all $n$ levels are passed, it might happen that the residue $r \neq 1$ at line \ref{alg:membership_return}, which also indicates that $g \notin G$. 

In the literature, \ref{alg:membership} is usually referred to as
\emph{sifting} or \emph{stripping} of the group element $g$.

\section{The main algorithm}
Finally, we can now present the Schreier-Sims algorithm itself. It
uses a more efficient method of reducing the number of Schreier
generators considered, by making use of \ref{alg:membership}. 

Using
the notation from Theorem \ref{thm_leon}, we want to show that
$H^i_{\alpha_{i + 1}} = H^{i + 1}$ for eavch $i$, or equivalently that
all Schreier generators for $H^i$ are in $H^{i + 1}$. If we proceed
from $i = n - 1, \dotsc, 0$ instead of the other way, then for $H^n = 1$ we obviously already have a base and strong generating set,
so we can use \ref{alg:membership} to check if the Schreier generators
for $H^{n - 1}_{\alpha_n}$ are in $H^n$.

When we have checked all Schreier generators for $H^{n - 1}_{\alpha_n}$ we then have a
base and strong generating set for $H^{n - 1}$ by Theorem \ref{thm_leon},
since $H^{n - 1}_{\alpha_n} = H^n$, and we can therefore proceed
inductively downwards. This is shown in \ref{alg:ss_main} and \ref{alg:ss}.

\begin{algorithm} 
\dontprintsemicolon
\caption{\texttt{ComputeBSGS}}
\SetKwFunction{GetPartialBSGS}{GetPartialBSGS}
\SetKwFunction{SchreierSims}{SchreierSims}
\SetKwData{SGS}{sgs}
\SetKwData{Base}{base}
\KwData{A group $G = \gen{S}$ acting on a finite set $X$.}
\KwResult{A base and strong generating set for $G$}.
\Begin
{
  $(\Base, \SGS) := \GetPartialBSGS(S, \emptyset)$ \;
  \For{$i := \abs{\Base}$ \KwTo $1$}
  {
    $(\Base, \SGS) := \SchreierSims(\Base, \SGS, i)$ \;
  }
  \Return{$(\Base, \SGS)$}
}
\refstepcounter{algorithm}
\label{alg:ss_main}
\end{algorithm}

\begin{algorithm} 
\dontprintsemicolon
\caption{\texttt{SchreierSims}}
\SetKwFunction{GetSchreierGenerator}{GetSchreierGenerator}
\SetKwFunction{ComputeSchreierTree}{ComputeSchreierTree}
\SetKwFunction{NewBasePoint}{NewBasePoint}
\SetKwFunction{Membership}{Membership}
\SetKwFunction{SchreierSims}{SchreierSims}
\SetKwData{Tree}{tree}
\SetKwData{Gens}{gens}
\SetKwData{Gen}{gen}
\SetKwData{Point}{point}
\SetKwData{Residue}{residue}
\SetKwData{DropOut}{dropout}

\KwData{A group $G$ acting on a finite set $X$, a partial base $B =
  (\alpha_1, \dotsc, \alpha_k)$ and corresponding partial strong
  generating set $S$ for $G$, an integer $1 \leq i \leq k$ such that
  $H^{j - 1}_{\alpha_j} = H^j$ for $j = i + 1, \dotsc, k$, and
  Schreier trees $T^{j - 1}$ for $\alpha_j^{H^{j - 1}}$ for $j = i + 1, \dotsc,
  k$.}

\KwResult{Possibly extended partial base $B = (\alpha_1, \dotsc,
  \alpha_m)$ and corresponding partial strong generating $S$ set for
  $G$ such that $H^{j - 1}_{\alpha_j} = H^j$ for $j = i, \dotsc,
  m$ and Schreier trees $T^{j - 1}$ for $\alpha_j^{H^{j - 1}}$ for $j = i,
  \dotsc, m$.}

\tcc{Assumes the existence of a function \texttt{NewBasePoint}$(g)$ that returns a point $p \in X$ such that $p^g \neq p$}
\Begin
{
  $\Gens := S^i$ \;
  $T^{i - 1} := \ComputeSchreierTree(\Gens, \alpha_i)$ \; \label{alg:ss_comp_tree}
  \ForEach{$p \in \alpha_i^{H^{i - 1}}$}
  {
    \ForEach{$s \in \Gens$}
    {
      $\Gen := \GetSchreierGenerator(T^{i - 1}, p, s)$ \;
      \If{$\Gen \neq 1$}
      {
        $(\Residue, \DropOut) := \Membership(T^i, \dotsc, T^k, \Gen)$ \;
        \If{$\Residue \neq 1$}
        {
          $S := S \cup \set{\Gen, \Gen^{-1}}$ \;
          \If{$\DropOut = k + 1$}
          {
            $\Point := \NewBasePoint(\Gen)$ \;
            $B := B \cup \set{\Point}$ \;
          }
          \For{$j := r$ \KwTo $i + 1$}
          {
            $(B, S) := \SchreierSims(B, S, j)$ \;
          }
        }
      }    
    }
  }
  \Return{$(B, S)$}
}
\refstepcounter{algorithm}
\label{alg:ss}
\end{algorithm}

\subsection{Matrix groups}
As can be seen in the given algorithms, the main part of the
Schreier-Sims algorithm is indepedent of the particular type of group,
but we are interested in the situation where $G \leq \GL(d, q)$ for
some $d \geq 1$ and some $q = p^r$ where $p$ is a prime number and $r
\geq 1$. Here, $d$ is called the \emph{degree} of $G$ and is the
number of rows (columns) of the matrices in $G$, and $q$ is the finite
field size, ie $G$ contains matrices over $\GF(q) = \F_q$. In this
setting, $G$ acts naturally and faithfully on the vector space $X = \F_q^d$ from the
right by multiplication of a row vector with a matrix. We denote the standard base in $F_q^d$ by $e_1, e_2, \dotsc, e_d$.

What remains to specify is the algorithm \texttt{NewBasePoint}, which
depends on the particular point set $X$ and the action that is used.
To construct this algorithm, we observe that given a matrix $M \neq I
\in G$, if $M_{ij} \neq 0$ for some $i \neq j$ then $e_i M \neq e_i$
so the row vector $e_i$ is a point moved by $M$. If $M$ instead is
diagonal, but not a scalar matrix, then $M_{ii} \neq M_{jj}$ for some
$i \neq j$, and $(e_i + e_j) M \neq e_i + e_j$ so this row vector is
moved by $M$. Finally if $M$ is scalar, then $M_{11} \neq 1$ since $M
\neq I$ and thus $e_1 M \neq e_1$. This translates directly into \ref{alg:new_base_point}.

\begin{algorithm} 
\dontprintsemicolon
\caption{\texttt{NewBasePoint}}
\KwData{A matrix $M \neq I \in G \leq \GL(d, q)$.}
\KwResult{A row vector $v \in \F_q^d$ such that $vM \neq v$}.
\SetKw{KwAnd}{and}
\Begin
{
  \For{$i := 1$ \KwTo $d$}
  {
    \For{$j := 1$ \KwTo $d$}
    {
      \If{$i \neq j$ \KwAnd $M_{ij} \neq 0$}
      {
        \Return{$e_i$}
      }
    }
  }
  \For{$i := 1$ \KwTo $d$}
  {
    \For{$j := 1$ \KwTo $d$}
    {
      \If{$i \neq j$ \KwAnd $M_{ii} \neq M_{jj}$}
      {
        \Return{$e_i + e_j$}
      }
    }
  }
  \Return{$e_1$}
}
\refstepcounter{algorithm}
\label{alg:new_base_point}
\end{algorithm}

\section{Complexity analysis}
We now analyse the time complexity of some of the given algorithms, but we
will not be interested in space complexity. For the analysis we assume
that the external functions \texttt{Tree}, \texttt{AddChild} and
\texttt{EdgeLabel}, that depend on the particular datastructure used,
take time $\Theta(1)$ This is a reasonable assumption and it is satisfied
in the code for the project, which uses hash tables to implement the Schreier trees. We also assume that the datastructure
used for sets is a sorted list, so that elements can be found, added and
removed in logarithmic time using binary search. This assumption is
satisfied in \GAP.

Moreover, since we are working in a matrix group $G = \gen{S} \leq
\GL(d, q)$ that acts on the vector space $X = \F_q^d$, we know that
the multiplication of two group elements takes time $\Theta(d^3)$ and
that the action of a group element on a point takes time
$\Theta(d^2)$, under the assumption that the multiplication of two
elements from $\F_q$ takes time $\Theta(1)$. This assumption is
reasonable since we are mostly interested in the case when $q$ is
small, so that one field element can be stored in a machine word and manipulated in constant time. Then
we also see that testing equality between two group elements takes
time $\OR{d^2}$ and testing equality between two points takes time
$\OR{d}$.

Consider first \ref{alg:orbit}. We know that it is a breadth-first
search and we know that a simple breadth-first search on a graph
$\mathcal{G} = (V, E)$ takes time $\OR{\abs{V} + \abs{E}}$. In our case
we have $\abs{E} = \abs{V} \abs{S} = \abs{X} \abs{S}$, so the edges
are dominating. We also see that line \ref{alg:orbit_innerloop1}
takes time $\Theta(d^2)$ and line \ref{alg:orbit_innerloop2} takes
time $\OR{\log{\abs{S}}}$ since the size of the set
\textsf{children} is bounded by $\abs{S}$. The former line is therefore dominating, and thus we have that
\ref{alg:orbit} takes time $\OR{\abs{X}\abs{S} d^2}$.

For \ref{alg:trace} we just note that in a tree with vertex set $V$,
the depth of any node is bounded by $\abs{V}$. In our case the
vertices of the Schreier tree is the points of the orbit, which may be
the whole of $X$ is the action is transitive. Therefore we see that
\ref{alg:trace} takes time $\OR{\abs{X} d^4}$.

It is obvious that \ref{alg:new_base_point} takes time $\OR{d^2}$, and
it is also obvious that \ref{alg:get_schreier_gen} takes time
$\OR{\abs{X} d^2 + d^3} = \OR{\abs{X} d^2}$. In \ref{alg:partial_bsgs}
we see that line \ref{alg:partial_bsgs_innerloop2} takes time $\OR{1}$
because the base is just a list and can therefore be augmented in
constant time. We see that line \ref{alg:partial_bsgs_innerloop1}
takes time $\OR{\abs{B} d^3}$ since we might have to multiply every
base point with the group element and check equality, in the case
where it actually fixes the base. Thus, \ref{alg:partial_bsgs} takes
time $\OR{\abs{S} \log{\abs{S}} \abs{B} d^3}$.

Since \ref{alg:pre_ss} is not used in the \GAP~package of the project,
we skip the complexity analysis of it and its related algorithms. We also skip any detailed complexity
of the Schreier-Sims algorithm itself, since we will see that for
matrix groups, this is futile anyway. The case of permutation groups is
analysed in \cite{butler91} and the result is that the Schreier-Sims
algorithm has time complexity $\OR{\abs{X}^5}$. 

The essential problem
when using the algorithm for matrix groups is then that $X = \F_q^d$
so $\abs{X} = q^d$ and the Schreier-Sims algorithm is thus exponential
in $d$. This is quite disheartening, since it is the case when $d$
grows large that we are interested in, more often than when $q$ is
large. However, it is possible to make the implementation fast enough for many practical purposes, so all is not lost.


