\documentclass[draft]{amsart}
\usepackage[english]{babel} 
\usepackage[latin1]{inputenc} % svenska tecken skall tolkas
\usepackage{amsmath}
\usepackage{amsfonts} % for \mathbb
\usepackage{amsthm}
\usepackage{amscd}
\usepackage{amsopn}
\usepackage{amstext}
\usepackage{amsxtra}
\usepackage{amsbsy}
\usepackage{bm} % for \boldsymbol
%\usepackage{graphicx}
\usepackage{draftcopy}
\frenchspacing

\title{The matrix Schreier-Sims algorithm in GAP}

\author{Henrik B\"a\"arnhielm}

\email{henrik.baarnhielm@imperial.ac.uk}
\urladdr{http://matrixss.sourceforge.net/}

\date{2004-02-08}

\theoremstyle{plain}
\newtheorem{theorem}{Theorem}[section]
\newtheorem{lemma}[theorem]{Lemma}
\newtheorem{cl}[theorem]{Corollary}
\newtheorem{pr}[theorem]{Proposition}
\newtheorem{axiom}[theorem]{Axiom}

\theoremstyle{definition}
\newtheorem{deff}[theorem]{Definition}

\theoremstyle{remark}
\newtheorem{note}{Note}
\newtheorem{remark}{Remark}

\providecommand{\abs}[1]{\left\lvert #1 \right\rvert}
\providecommand{\norm}[1]{\left\lVert #1 \right\rVert}
\providecommand{\ceil}[1]{\left\lceil #1 \right\rceil}
\providecommand{\floor}[1]{\lfloor #1 \rfloor}
\providecommand{\set}[1]{\left\lbrace #1 \right\rbrace}
\providecommand{\gen}[1]{\left\langle #1 \right\rangle}
\providecommand{\ord}[1]{\text{ord}( #1 )}

\newcommand{\field}[1]{\mathbb{#1}}
\newcommand{\vect}[1]{\boldsymbol{\mathrm{#1}}}
\newcommand{\N}{\field{N}}
\newcommand{\Z}{\field{Z}}
\newcommand{\R}{\field{R}}
\newcommand{\Q}{\field{Q}}
\newcommand{\K}{\field{K}}
\newcommand{\A}{\field{A}}

% for Cayley graphs
\newcommand{\C}{\mathcal{C}}

\DeclareMathOperator{\sgd}{sgd}
\DeclareMathOperator{\mgm}{mgm}
\DeclareMathOperator{\sgn}{sgn}
\DeclareMathOperator{\GL}{Gl}
\DeclareMathOperator{\IM}{Im}
\DeclareMathOperator{\RE}{Re}
\DeclareMathOperator{\I}{Id}

\begin{document}
\begin{titlepage}
\begin{abstract}
Bla bla
\end{abstract}


\maketitle
\thispagestyle{empty}

\end{titlepage}

\tableofcontents

\newpage

\section{Preface}

\section{Introduction} \label{intro}

\section{Preliminaries}
We begin with some elementary definitions. First some graph theory, where we follow \cite{biggs89}.

\begin{deff} \label{def_graph}
A \emph{graph} is an ordered pair $G = (V, E)$ where $V$ is a finite non-empty set, the \emph{vertices} of $G$ and $E \subseteq V \times V$ is the \emph{edges} of $G$.
\end{deff}
\begin{remark}
  A graph in the sense of \ref{def_graph} is sometimes called a
  \emph{combinatorial graph} in the literature, to emphasize that they
  are not \emph{metric graphs} in the sense of \cite{bridson99}. As we
  are interested only in finite graphs and not in geometry, we do not
  make use of this nomenclature.
\end{remark}

\begin{remark}
The definition implies that our graphs are directed, have no multiple edges, but may have loops.
\end{remark}

\begin{deff} \label{def_graph_misc}
Let $G = (V, E)$ is a graph, a sequence $v = v_1, v_2, \dotsc, v_n =
  u$ of vertices of $G$ such that $(v_i, v_{i + 1}) \in E$ for $i = 1,
  \dotsc, n - 1$ is called a \emph{walk} from $v$ to $u$. If $v = u$
  then the walk is called a \emph{cycle}. If all vertices in the walk are
  distinct, it is called a \emph{path}. If every pair of distinct vertices in $G$ can be joined by a path, then $G$ is \emph{connected}.
\end{deff}

\begin{deff}
A graph $G^{\prime} = (V^{\prime}, E^{\prime})$ is a \emph{subgraph} of a graph $G = (V, E)$ if $V^{\prime} \subseteq V$ and $E^{\prime} \subseteq E$.
\end{deff}

\begin{deff} \label{def_tree}
A \emph{tree} is a connected graph without cycles. A \emph{rooted tree} is a tree $T = (V, E)$ where a vertex $r \in V$ have been specified as the \emph{root}.
\end{deff}

\begin{deff}
If $G = (V, E)$ is a graph, then a tree $T$ is a \emph{spanning tree} of $G$ if $T$ is a subgraph of $G$ and $T$ have vertex set $V$.
\end{deff}

The following are elementary and the proofs are omitted.
\begin{pr}
In a tree $T = (V, E)$ we have $\abs{E} = \abs{V} - 1$ and there exists a \emph{unique} path between every pair of distinct vertices. Moreover, if $G$ is a graph where every pair of distinct vertices can be joined by a unique path, then $G$ is a tree.
\end{pr}

\begin{pr}
Every graph has a spanning tree.
\end{pr}

Now some group theory.
\begin{deff}
If $G = \gen{S}$ is a finite group, then the \emph{Cayley graph} $\C_G(S)$ is the graph with vertex set $G$ and edges $E = \set{ (g, sg) \mid g \in G, s \in S}$.
\end{deff}

\bibliographystyle{amsalpha} 
\bibliography{schreiersims}

\end{document}
