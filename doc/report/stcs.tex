\chapter{The Schreier-Todd-Coxeter-Sims algorithm}

Coset enumeration is one of the oldest algorithms in group theory, and
it was first described in \cite{tc36}. Given a finitely presented
group $G$ and $H \leq G$, the aim of coset enumeration is to construct
the list of the cosets of $H$ in $G$, ie to find a set $X$ with a
point $p \in X$ such that $G$ acts transitively on $X$ and such that
$H = G_p$. The set $X$ is usually taken to be $\set{1, \dotsc, n}$ for
some $n$ and $p = 1$. If the index of $H$ is not finite, then the algorithm will usually not terminate.

The coset enumeration is a trial-and-error process, and quite a few
strategies have been developed over the years. We will not be
concerned about the actual algorithm, since it is beyond the scope of this report, and there is a good
implementation in \GAP, which is used in this project. A more detailed
description of coset enumeration can be found in \cite{sims98}.

However, it is important to be aware of the overall structure of coset
enumeration. It works by letting the elements of $G$ act on $X$ and
introducing new cosets when necessary. Then it may happen that some
cosets turn out to be the same and thus are identified in the list.
The structure of the algorithm makes it possible to exit prematurely
when a given number of cosets have been defined, and this can be used
if one knows an upper bound on the number of cosets.

In our case, we want to use the last case of Theorem \ref{thm_leon}.
At level $i$ of Schreier-Sims algorithm, we know that the base and SGS
are complete for higher levels, and want to verify that $H^{i + 1} =
H^i_{\alpha_{i + 1}}$, which by the theorem is equivalent to $[H^i :
H^{i + 1}] = \abs{\alpha_{i + 1}^{H^i}}$ and since we can compute a
Schreier tree for $\alpha_{i + 1}^{H^i}$ we know the orbit size. Thus,
if we can compute $[H^i : H^{i + 1}]$ using coset enumeration and it turns out to be equal the
orbit size, it is unnecessary to compute any Schreier generators and
check for membership.

All this is formalised in \ref{alg:stcs} where we use the option to
exit the coset enumeration after $M \abs{\alpha_{i + 1}^{H^i}}$ cosets have
been defined, for some rational number $M \geq 1$, since this
certainly is an upper bound. In our implementation, the number $M$ can
be specified by the user, and the default is $M = 6/5$. This
value comes from \cite{leon80}, where the algorithm was introduced and
some experimentation was carried out to determine a good value of $M$.

\begin{algorithm} 
\dontprintsemicolon
\caption{\texttt{SchreierToddCoxeterSims}}
\SetKwFunction{GetSchreierGenerator}{GetSchreierGenerator}
\SetKwFunction{ComputeSchreierTree}{ComputeSchreierTree}
\SetKwFunction{NewBasePoint}{NewBasePoint}
\SetKwFunction{Membership}{Membership}
\SetKwFunction{ToddCoxeter}{ToddCoxeter}
\SetKwFunction{SchreierToddCoxeterSims}{SchreierToddCoxeterSims}
\SetKwData{Tree}{tree}
\SetKwData{Gens}{gens}
\SetKwData{Gen}{gen}
\SetKwData{Point}{point}
\SetKwData{Table}{table}
\SetKwData{Residue}{residue}
\SetKwData{DropOut}{dropout}

\KwData{A group $G$ acting on a finite set $X$, a partial base $B =
  (\alpha_1, \dotsc, \alpha_k)$ and corresponding partial strong
  generating set $S$ for $G$, an integer $1 \leq i \leq k$ such that
  $H^{j - 1}_{\alpha_j} = H^j$ for $j = i + 1, \dotsc, k$, and
  Schreier trees $T^{j - 1}$ for $\alpha_j^{H^{j - 1}}$ for $j = i + 1, \dotsc,
  k$.}

\KwResult{Possibly extended partial base $B = (\alpha_1, \dotsc,
  \alpha_m)$ and corresponding partial strong generating $S$ set for
  $G$ such that $H^{j - 1}_{\alpha_j} = H^j$ for $j = i, \dotsc,
  m$ and Schreier trees $T^{j - 1}$ for $\alpha_j^{H^{j - 1}}$ for $j = i,
  \dotsc, m$.}

\tcc{Assumes the existence of a function \texttt{NewBasePoint}$(g)$ that returns a point $p \in X$ such that $p^g \neq p$.}
\tcc{Assumes the existence of a function \texttt{ToddCoxeter}$(U_1, U_2, k)$ that performs coset enumeration on $G = \gen{U_1}$ and $H = \gen{U_2} \leq G$, exiting when $k$ cosets have been defined.}
\Begin
{
  $\Gens := S^i$ \;
  $T^{i - 1} := \ComputeSchreierTree(\Gens, \alpha_i)$ \;
  \ForEach{$p \in \alpha_i^{H^{i - 1}}$}
  {
    \ForEach{$s \in \Gens$}
    {
      $\Table := \ToddCoxeter(S^i, S^{i + 1}, \abs{T^i} + 1)$ \;
      \If{$\abs{\Table} = \abs{T^i}$}
      {
        \Return{$(B, S)$}
      }
      $\Gen := \GetSchreierGenerator(T^{i - 1}, p, s)$ \;
      \If{$\Gen \neq 1$}
      {
        $(\Residue, \DropOut) := \Membership(T^i, \dotsc, T^k, \Gen)$ \;
        \If{$\Residue \neq 1$}
        {
          $S := S \cup \set{\Gen, \Gen^{-1}}$ \;
          \If{$\DropOut = k + 1$}
          {
            $\Point := \NewBasePoint(\Gen)$ \;
            $B := B \cup \set{\Point}$ \;
          }
          \For{$j := r$ \KwTo $i + 1$}
          {
            $(B, S) := \SchreierToddCoxeterSims(B, S, j)$ \;
          }
        }
      }    
    }
  }
  \Return{$(B, S)$}
}
\refstepcounter{algorithm}
\label{alg:stcs}
\end{algorithm}

The Schreier-Todd-Coxeter-Sims algorithm is known to perform
particularly good when the input partial base and SGS are actually
already complete. Thus, it can be used for \emph{verification} of the output
from a probabilistic algorithm, and in this project it is mainly used
in that way, to verify the output from the probabilistic algorithm
described earlier.
