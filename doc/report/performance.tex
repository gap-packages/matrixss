\chapter{Performance}

As observed earlier, one of the objectives of the project was to make
an implementation that hopefully would be faster than the one already
existing in \GAP. To determine if this objective was met, the
implementation has been benchmarked and compared with the built-in
\GAP~implementation.

The algorithm has been used to compute a base and SGS for some matrix
groups that are easy to construct in \GAP, and the generating sets
that were used were the standard generating sets from the
\GAP~library. The main test groups were classical groups: the general
and special linear groups $\GL(d, q)$ and $\SL(d, q)$ and the general
and special orthogonal groups $\GO(d, q)$ and $\SO(d, q)$, for various
(small) $d$ and $q$. The algorithm were also tested on some Suzuki
groups $\Sz(q)$ (where $q$ is a non-square power of $2$) and some Ree
groups $\Ree(q)$ (where $q = 3^{1 + 2m}$ for some $m > 0$).

The benchmark was carried out on quite standard PC, with an AMD Athlon
CPU running at $2$ GHz and with $1$ GB of RAM. It is likely
that these were the only important parameters, since \GAP~is mainly
CPU intensive, and there was enough of RAM to avoid any
swapping, so the hard disk speed should be a negligible factor. The
\GAP~installation tests gave a GAP4stones value of $194624$ and the computer
was running the RedHat Linux operating system. 

The details of the benchmark is shown in the appendix. For small groups, the existing algorithm was the faster one, but as the matrix and field size increased, our implementation became the faster one, and this indicates that the time complexity of our implementation is actually better. 

In terms of memory, our implementation also seems to be much better.
No rigorous memory benchmark has been performed, but during the
benchmark, a simple inspection of the memory usage by the \GAP~process
was carried out. This indicated that our implementation used about half the memory needed by the existing implementation, and when some tests with even larger field and matrix sizes was performed, the difference seemed even larger, since \GAP~actually ran out of memory when executing the existing implementation.

The conclusion must therefore be that the project is quite a success.
