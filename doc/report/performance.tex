\chapter{Performance}

As observed earlier, one of the objectives of the project was to make
an implementation that hopefully would be faster than the one already
existing in \GAP. To determine if this objective was met, the
implementation has been benchmarked and compared with the built-in
\GAP~implementation.

The algorithm has been used to compute a base and SGS for some matrix
groups that are easy to construct in \GAP, and the generating sets
that were used were the standard generating sets from the
\GAP~library. The main test groups were classical groups: the general
and special linear groups $\GL(d, q)$ and $\SL(d, q)$ and the general
and special orthogonal groups $\GO(d, q)$ and $\SO(d, q)$, for various
(small) $d$ and $q$. The algorithm were also tested on some Suzuki
groups $\Sz(q)$ (where $q$ is a non-square power of $2$) and some Ree
groups $\Ree(q)$ (where $q = 3^{1 + 2m}$ for some $m > 0$).

The benchmark was carried out on quite standard PC, with an Intel Celeron
CPU running at $1,7$ GHz and with $128$ MB of RAM. It is likely
that these were the only important parameters, since \GAP~is mainly
CPU intensive, and there seemed to be enough of RAM to avoid any
swapping, so the hard disk speed should be a negligible factor. The computer
was running the Debian GNU/Linux operating system. 

The details of the benchmark is shown in the appendix. In general, it
turned out that the existing algorithm was the faster one, but
for some groups, the Suzuki groups for instance, our implementation
was much faster. The conclusion must therefore be that the project actually
is a success, however small.
