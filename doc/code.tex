% This file was created automatically from code.msk.
% DO NOT EDIT!
%%%%%%%%%%%%%%%%%%%%%%%%%%%%%%%%%%%%%%%%%%%%%%%%%%%%%%%%%%%%%%%%%%%%%%%%%%%%%%%
%%
%W    code.msk     The Matrix Schreier Sims package - Documentation
%%
%H    File      : $RCSfile$
%H    Author    : Henrik Bäärnhielm
%H    Dev start : 2004-07-28 
%%
%H    Version   : $Revision$
%H    Date      : $Date$
%H    Last edit : $Author$
%%
%H    @(#)$Id$
%%
%%%%%%%%%%%%%%%%%%%%%%%%%%%%%%%%%%%%%%%%%%%%%%%%%%%%%%%%%%%%%%%%%%%%%%%%%%%%%%%

\Chapter{Implementation}

Here is the documentation from the package source code. As the package
works with matrix groups acting on vector spaces, references to <group
elements> means <matrices>, ie a list of row vectors, each of which is
list of field elements. References to <points> means <row vectors>, ie
elements of the vector space on which the group acts.

The code tries to avoid the computation of inverse matrices as much as
possible, and to accomplish this, the inverse of a group element is
stored together with the element in a list of length 2. Each time some
computation is made with the element, a similar computation is made
with the inverse, so that they are kept consistent. Therefore, in many
cases in the code, <group element> means an immutable list of 2
matrices that are inverses to each other.

%%%%%%%%%%%%%%%%%%%%%%%%%%%%%%%%%%%%%%%%%%%%%%%%%%%%%%%%%%%%%%%%%%%%%%%%%%%%%%%
\Section{General code}





These are the general declarations used by the package. Most notably the
attribute `StabChainMatrixGroup' which is the core of the package 
functionality.




\>`StabChainMatrixGroup( G )'{StabChainMatrixGroup}!{general} A

Declare new attribute for storing base, SGS and Schreier trees.
The attribute is computed using the Schreier-Sims algorithm for finite
matrix groups, which is the main content of the package.

The attribute is a record with two components:
\beginitems
`SchreierStructure' & the main information structure, see "ssInfo".

`SGS' & a list of the strong generators
\enditems

The corresponding attribute operations are aware of a few Options.
\beginitems
`SimpleSchreierTree' & calculate coset representatives at the moment of
                    creation of the Schreier trees, thus making them
                    have height 1.
                    This should make the algorithm significantly 
                    faster.

`ExtendSchreierTree' & Do not recompute Schreier trees at each run of a
                      given level, but extend the Schreier trees from 
                      the last run at that level.

`AlternatingActions' & Always prepend a base point with the line that
                      contains it, using the projective action on the 
                      line.

`CleverBasePoints' & Choose an initial list of base points using 
                    `BasisVectorsForMatrixAction', which is made by
                    O'Brien and Murray.

`ShallowSchreierTree' & Create the Schreier trees using the method described
                       by Babai et al (1991) which guarantees logarithmic
                       depth. This Option is only used when the option
                       `SimpleSchreierTree' is <not> defined.

`Random' & Use probabilistic algorithm.

`Linear' & Use nearly linear-time algorithm. This takes precedence over
          `Random', if it is present.

\enditems




\>`ssInfo' V

Main structure holding information for the algorithm. This is not a global
variable, but the same structure is used in all the variants of the 
algorithm, but all members are not necessarily used.

The structure `ssInfo' is a list of records, with a record for each level 
in the algorithm, ie one record for each base point. New base points
may of course be added to the base during the execution of the algorithm,
and then a new record is added to the end of the list.

The members of the record are:
\beginitems
`partialSGS' & the elements in the current partial SGS that fixes all
            points at lower levels, or the whole partial SGS for the
            first level

`partialBase' & the base point for this level

`action' & the action (function) at this level

`points' & the field where the base point `partialBase' comes from

`hash' & the hash function for the Schreier tree at this level

`schreierTree' & the Schreier tree for this level, representing the
                basic orbit at this level, ie the orbit of `partialBase'
                under the action of `partialSGS' at the previous (lower) 
                level. Thus, the root of the tree is `partialBase'.

`oldSGS' & the whole partial SGS at the last call of SchreierSims at
          this level

`IsIdentity' & the function to check if a point is the identity at this
              level
  



\>`MatrixSchreierSimsInfo' V

The {\GAP} InfoClass used by the package, for debugging purposes.




\>`MATRIXSS_DEBUGLEVEL'{MATRIXSS_DEBUGLEVEL} V

The internal debugging level. This is really obsolete and the above info
class should be used instead.




\>`MATRIXSS_BasePointStore' V

A list of hopefully good base points, ie base points with small orbits.
They are fetched with `BasisVectorsForMatrixAction' which is due to
O{\rq}Brien and Murray, and as long as the list is non-empty, 
new base points will be shifted from it.




\>`MatrixGroupOrderStabChain( ssInfo )'{MatrixGroupOrderStabChain![code.gd]}@{`MatrixGroupOrderStabChain'!`[code.gd]'} F

Computes the order of the group defined by the given Schreier trees, see
"ssInfo".








These are the common functions used by all the variants of the Schreier-Sims
algorithm implemented in the package.




These are the core functions of the package.

\>`Size( G )'{Size}!{finite matrix group} A

A method for `Size' for finite matrix groups, that uses the implementation 
of Schreier-Sims algorithm in this package, ie it uses the 
`StabChainMatrixGroup' attribute to compute the order of `G'.

This method is only installed if `MATRIXSS_TEST' is not defined when the 
package is loaded.




\>MATRIXSS_GetPartialBaseSGS( generators, identity, field ) F

Constructs a partial base and a partial SGS given a set of generators
for a group. Returns the partial SGS and the `ssInfo' structure, 
see "ssInfo".
\beginitems
`generators' & given set of generators

`identity'   & group identity element (the identity matrix)

`field'      & the vector space on which the group acts
\enditems




\>`MATRIXSS_Membership( ssInfo, element, identity )'{MATRIXSS_Membership} F

Check if an element belongs to a group, using sifting
\beginitems
`ssInfo' & Main information structure about our stabiliser chain. The Schreier
        trees is used during the sifting.

`element' & the element to check for membership 

`identity' & group identity
\enditems




\>MATRIXSS_NewBasePoint( element, identity, field ) F

Find a point not in base that is moved by the given element 
(which fixes the base)
\beginitems
`element' & the bad element that fixes the whole base

`identity' & the group identity (the identity matrix)

`field'    & the vector space on which the group acts
\enditems




\>`MATRIXSS_GetSchreierGenerator( schreierTree, generator, point, action, identity, IsIdentity )'{MATRIXSS_GetSchreierGenerator} F

Creates a Schreier generator for the stabiliser in the group which has 
`generator' as one of its generators. The stabiliser fixes `point' under
`action'.




\>MATRIXSS_ExtendBase( ssInfo, badElement, identity ) F

Add a new base point to the base, so that the given element is not in the
stabiliser of the point
\beginitems
`ssInfo' & main information structure for the current Schreier-Sims run

`badElement' & the element that fixes all current base points

`identity' & the group identity
\enditems




\>MATRIXSS_AugmentBase( ssInfo, newPoint, action, hash, identity ) F

Add a new base point to the base, so that the given element is not in the
stabiliser of the point
\beginitems
`ssInfo' & main information structure for the current Schreier-Sims run

`newPoint' & the point to add to the base

`action' & the action for the new point

`hash' & the dictionary info for the new point

`identity' & the group identity
\enditems




\>`MATRIXSS_OrbitElement( schreierTree, point, action, identity, IsIdentity )'{MATRIXSS_OrbitElement} F

Compute the group element that connects the root of the Schreier tree to
a given point. This function assumes that the point actually is in the 
orbit described by the given Schreier tree.
\beginitems
`schreierTree' & Schreier tree for the orbit to use

`point' & the point to check if it is in the orbit

`action' & the action that was used to create the Schreier tree

`identity' & the group identity (the identity matrix)

`IsIdentity' & function to use when checking if a group element is equal
              to the identity
\enditems




\>MATRIXSS_ComputeSchreierTree( tree, generators, action, root, hash, identity ) F

Fill a Schreier tree that contains only the root.
\beginitems
`tree' & The Schreier tree to fill.

`generators' & The generators for the group that gives rise to the orbit
              represented by the Schreier tree.

`action' & The action of the group on the point set.

`root' & The root point of the tree.

`hash' & The dictionary info for the tree, used to create hash function.

`identity' & the group identity (the identity matrix)

\enditems




\>MATRIXSS_ExtendSchreierTree( oldTree, generators, oldGenerators, action, dictinfo ) F

Extends an existing Schreier tree by a given set of generators
\beginitems
`oldTree' & The Schreier tree to extend, ie a Dictionary.

`generators' & The generators for the group that gives rise to the orbit
              represented by the Schreier tree.

`oldGenerators' & The current generators (edge-labels) of `oldTree'.

`action' & The action of the group on the point set.

`dictinfo' & The Dictionary info used when `oldTree' was created.
\enditems




\>`MATRIXSS_OrbitElement_ToddCoxeter( schreierTree, point, action, identity, IsIdentity, freeGroup, genMap )'{MATRIXSS_OrbitElement_ToddCoxeter} F

Special version of `MATRIXSS_OrbitElement', see "MATRIXSS_OrbitElement", 
that also calculates the Word in the generators of the group element it 
returns.

More specifically, it computes the Word of the generators of the 
corresponding free group.
\beginitems
`schreierTree' & Schreier tree for the orbit to use

`point' & the point to check if it is in the orbit

`action' & the action that was used to create the Schreier tree

`identity' & the group identity (the identity matrix)

`IsIdentity' & function to use when checking if a group element is equal
              to the identity

`freeGroup' & corresponding free group to the group whose generators form
             the set of edge labels of `schreierTree'

`genMap' & list of 2 lists of the same length, the first being the edge 
          labels of `schreierTree' (the generators of the corresponding 
          group), and the second being the corresponding generators of
          `freeGroup'
\enditems




\>MATRIXSS_Membership_ToddCoxeter( ssInfo, element, identity, freeGroup ) F

Special version of `MATRIXSS_Membership', see "MATRIXSS_Membership", that 
also expresses the sifted group element as a word in the generators of a 
given free group.
\beginitems
`ssInfo' & Main information structure about our stabiliser chain. 
          The Schreier trees is used during the sifting.

`element' & the element to check for membership 

`identity' & group identity

`freeGroup' & the free group in which the sifted element will be expressed
\enditems




\>MATRIXSS_GetSchreierGenerator_ToddCoxeter( schreierTree, generator, point, action, identity, IsIdentity, freeGroup, genMap ) F

Special version of `MATRIXSS_GetSchreierGenerator', see 
"MATRIXSS_GetSchreierGenerator", that also returns the Schreier generator
as a Word in the generators of `freeGroup', using `genMap' to map the
generators to the free group. See "MATRIXSS_OrbitElement_ToddCoxeter".




\>MATRIXSS_CreateShallowSchreierTree( orbitTree, root, generators, labels, action, identity, hash ) F

Create a shallow Schreier tree, ie with at most logarithmic height.
\beginitems
`orbitTree' & Given tree representing the same orbit as the shallow Schreier
             to be computed.

`root' & The root point of the tree.

`generators' & From this set will any new edge labels be taken.

`labels' & The elements that, together with its inverses, will form the
            edge labels of the tree.

`action' & The action of the group on the point set.

`identity' & the group identity (the identity matrix)

`hash' & The dictionary info for the tree, used to create hash function.

\enditems




\>MATRIXSS_GetSchreierTree( oldTree, root, generators, oldGenerators, action, hash, identity ) F

Returns a Schreier tree. This routine encapsulates the other Schreier tree
functions.
\beginitems
`oldTree' & The Schreier tree to extend, in case there should be an 
           extension.

`root' & The root point of the tree.

`generators' & The generators for the group that gives rise to the orbit
              represented by the Schreier tree.

`oldGenerators' & The current generators (edge-labels) of `oldTree'.

`action' & The action of the group on the point set.

`hash' & The dictionary info for the tree, used to create hash function.

`identity' & the group identity (the identity matrix)

\enditems




\>MATRIXSS_MonotoneTree( root, elements, action, identity, dictinfo ) F

Create a monotone Schreier tree with given root and edge labels.
\beginitems
`root' & The root point of the tree.

`elements' & The elements that, together with its inverses, will form the
            edge labels of the tree.

`action' & The action of the group on the point set.

`identity' & the group identity (the identity matrix)

`dictinfo' & The dictionary info for the tree, used to create hash function.

\enditems




\>MATRIXSS_RandomCosetRepresentative( schreierTree, action, identity ) F

Return a random coset representative from the transversal defined by 
`schreierTree'.




\>MATRIXSS_RandomOrbitPoint( schreierTree ) F

Returns a random point in the orbit given by `schreierTree'.




\>MATRIXSS_RandomSchreierGenerator( schreierTree, elements, action, identity ) F

Return a random Schreier generator constructed from the points in 
`schreierTree' and the generators in `elements'.




\>MATRIXSS_RandomSubproduct( elements, identity ) F

Return a random subproduct of `elements'.




These are also core function, but of slightly less importance, or mainly of technical nature.

\>`MATRIXSS_SubProdGroups' V

A Dictionary of SymmetricGroups, used when permuting random subproducts.




\>MATRIXSS_CopySchreierTree( tree, dictinfo ) F

Creates a copy of a whole Schreier tree, ie of makes a copy of the 
Dictionary.
\beginitems
`tree' & the Dictionary to copy

`dictinfo' & the dictinfo that was used when creating `tree'
\enditems




\>MATRIXSS_GetOrbitSize( schreierTree ) F

Get size of orbit defined by the given Schreier tree.




\>MATRIXSS_GetOrbit( schreierTree ) F

Return all points (as a list) in the orbit of the point which is root of 
the Schreier tree, ie return all keys in the Dictionary.
The list is not necessarily sorted, and it is mutable.




\>MATRIXSS_IsPointInOrbit( schreierTree, point ) F

Check if the given point is in the orbit defined by the given Schreier tree.




\>MATRIXSS_CreateInitialSchreierTree( root, dictinfo, identity ) F

Create a Schreier tree containing only the root.
\beginitems
`root' & The base point that is to be the root of the Schreier tree.

`dictinfo' & Used when creating the Dictionary that is the Schreier tree

`identity' & the group identity element
\enditems




\>MATRIXSS_GetSchreierTreeEdge( schreierTree, point ) F

Get the label of the edge originating at the given point, and directed 
towards the root of the given Schreier tree.




\>MATRIXSS_ProjectiveIsIdentity( element, identity ) F

Identity check when using projective action (all scalar matrices are
considered equal to the identity)
\beginitems
`element' & the group element to check if it is equal to identity

`identity' & the group identity (the identity matrix)
\enditems




\>MATRIXSS_IsIdentity( element, identity ) F

Identity check when using normal point action 
\beginitems
`element' & the group element to check if it is equal to identity

`identity' & the group identity (the identity matrix)
\enditems




\>MATRIXSS_PointAction( point, element ) F

The action of a group element (a matrix) on a point (a row vector).
The action is from the right 
\beginitems
`point' & The point (row vector) to act on.

`element' & The group element (matrix) that acts.
\enditems




\>MATRIXSS_ProjectiveAction( point, element ) F

The projective action of a matrix on a row vector.
The one-dimensional subspace corresponding to the point is represented
by the corresponding normed row vector
\beginitems
`point' & The point to act on. Must be a *normed* row vector.

`element' & The group element (matrix) that acts.
\enditems




\>MATRIXSS_DebugPrint( level, message ) F

Internal function for printing debug messages. Uses the internal variable
`MATRIXSS_DEBUGLEVEL', see "MATRIXSS_DEBUGLEVEL", to determine if the 
message should be printed.





%%%%%%%%%%%%%%%%%%%%%%%%%%%%%%%%%%%%%%%%%%%%%%%%%%%%%%%%%%%%%%%%%%%%%%%%%%%%%%%
\Section{Deterministic algorithm}





These are the special routines for the deterministic version of 
Schreier-Sims algorithm.



\>`StabChainMatrixGroup( G )'{StabChainMatrixGroup}!{deterministic} A

An implementation of the Schreier-Sims algorithm, for matrix groups,
probabilistic version. See "StabChainMatrixGroup!general" for general information
about the attribute.




\>SchreierSims( ssInfo, partialSGS, level, identity ) F

The main Schreier-Sims function, which is called for each level.
\beginitems    
`ssInfo' & main information structure for the current Schreier-Sims run
  
`partialSGS' & given partial strong generating set
  
`level' & the level of the call to Schreier-Sims

`identity' & the group identity
\enditems




%%%%%%%%%%%%%%%%%%%%%%%%%%%%%%%%%%%%%%%%%%%%%%%%%%%%%%%%%%%%%%%%%%%%%%%%%%%%%%%
\Section{Probabilistic algorithm}





These are the special routines for the probabilistic implementation of 
Schreier-Sims algorithm.




\>`StabChainMatrixGroup( G )'{StabChainMatrixGroup}!{probabilistic} A

An implementation of the Schreier-Sims algorithm, for matrix groups,
probabilistic version. See "StabChainMatrixGroup!general" for general information
about the attribute.

In addition to the general Options of the attribute `StabChainMatrixGroup',
the probabilistic algorithm is aware of the following:
\beginitems
`Probability' & (lower bound for) probability of correct solution, which
                  defaults to 3/4

`Verify' & Boolean parameter which signifies if the base and SGS computed
        using the random Schreir-Sims algorithm should be verified
        using the Schreier-Todd-Coxeter-Sims algorithm.
        Defaults to `false.'

`OrderLowerBound' & Lower bound for the order of `G', must be $\geq$ 1.
                 Defaults to 1.

`OrderUpperBound' & Upper bound for the order of `G', or 0 if unknown.
                 Defaults to 0.

Note that if the order of `G' is known, so that 
`OrderLowerBound' = `OrderUpperBound' = `Size(G)'
then the randomized algorithm always produces a correct base and SGS, so
there is no need of verification. Also, the verification will extend the
given base and SGS to a complete base and SGS if needed.




\>RandomSchreierSims( ssInfo, partialSGS, maxIdentitySifts, identity, low_order, high_order ) F

The main random Schreier-Sims function.
\beginitems    
`ssInfo' & main information structure for the Schreier-Sims 
  
`partialSGS' & given partial strong generating set

`maxIdentitySifts' & maximum number of consecutive elements that sifts to 
                  identity before the algorithm terminates
  
`identity' & the group identity
  
`lowOrder' & lower bound on the group order (must be $\geq$ 1)
  
`highOrder' & upper bound on the group order, or 0 if not available
\enditems




%%%%%%%%%%%%%%%%%%%%%%%%%%%%%%%%%%%%%%%%%%%%%%%%%%%%%%%%%%%%%%%%%%%%%%%%%%%%%%%
\Section{STCS algorithm}





These are the Schreier-Todd-Coxeter-Sims routines, ie Schreier-Sims 
algorithm with additional calls to Todd-Coxeter coset enumeration to
possibly speed up the process. It is known to be fast when the input is
already a base and SGS, and therefore it is good for verifying a proposed
base and SGS, for example the output of a probabilistic algorithm.




\>MATRIXSS_SchreierToddCoxeterSims( ssInfo, partialSGS, level, identity, cosetFactor ) F

The main function for the Schreier-Todd-Coxeter-Sims algorithm. It is very
similar to ordinary Schreier-Sims algorithm and has a similar interface.
\beginitems    
`ssInfo' & main information structure for the current Schreier-Sims run
  
`partialSGS' & given partial strong generating set
  
`level' & the level of the call to Schreier-Sims

`identity' & the group identity

`cosetFactor' & the quotient of the maximum number of cosets generated 
                during coset enumeration and the corresponding orbit size
\enditems




%%%%%%%%%%%%%%%%%%%%%%%%%%%%%%%%%%%%%%%%%%%%%%%%%%%%%%%%%%%%%%%%%%%%%%%%%%%%%%%
\Section{Nearly linear time algorithm}





These are the special routines for the nearly linear-time version, described
in Babai et al, 1991.



\>`StabChainMatrixGroup( G )'{StabChainMatrixGroup}!{nearly linear time} A

An implementation of the Schreier-Sims algorithm, for matrix groups.
This version is inspired by the nearly linear time algorithm, described in 
\cite{seress03}. See "StabChainMatrixGroup!general" for general information
about the attribute.




\>ConstructSGS( ssInfo, partialSGS, identity ) F

The main Schreier-Sims function for the nearly linear-time algorithm.
\beginitems    
`ssInfo' & main information structure for the current Schreier-Sims run
  
`partialSGS' & given partial strong generating set
  
`identity' & the group identity
\enditems




\>CompletePointStabiliserSubgroup( ssInfo, element, level, identity, maxIdentitySifts ) F

The work-horse of the nearly linear-time algorithm, called for each level.
\beginitems    
`ssInfo' & main information structure for the current Schreier-Sims run
  
`element' & the element which

`partialSGS' & given partial strong generating set
  
`identity' & the group identity
\enditems




%%%%%%%%%%%%%%%%%%%%%%%%%%%%%%%%%%%%%%%%%%%%%%%%%%%%%%%%%%%%%%%%%%%%%%%%%%%%%%%
\Section{Verify routine}



\>MatrixSchreierSimsVerify( ssInfo, SGS, identity ) F

The <Verify> routine by Sims. Checks whether the given `ssInfo' and `SGS' 
encodes a base and strong generating set, and returns a record with 
components `Residue' and `Level'. In case the verification succeeds, the
level is 0 and the residue is the identity. Otherwise the residue is an
element that is in the stabiliser of the group at the indicated level, but
is not in the group at the next higher level.

\beginitems    
`ssInfo' & proposed structure to check
  
`SGS' & proposed SGS to check
  
`identity' & the group identity
\enditems




\>MATRIXSS_VerifyLevel( ssInfo, partialSGS, level, identity ) F

Checks that the stabiliser of the group at the given level is the same as
the group at the next higher level.

\beginitems    
`ssInfo' & proposed structure to check
  
`partialSGS' & proposed SGS to check
  
`level' & level to check

`identity' & the group identity
\enditems




\>MATRIXSS_VerifyMultipleGenerators( generators, schreierTree, point, action, hash, subGenerators, ssInfo, SGS, identity, points, IsIdentity, field ) F

Verifies that the stabiliser of the group generated by `generators', at the
point `point' is the same group as the group generated by `generators' minus
`subGenerators'. If so, the identity is returned, and otherwise an element
that is in the difference is returned.

`ssInfo' and `SGS' should be a base and strong generating set for the 
smaller group.

\beginitems    
`generators' & generators of the bigger group
  
`schreierTree' & Schreier tree for the orbit of `point' under `action' of 
                the group generated by `generators'
  
`point' & the point to get the stabiliser of

`action' & the action to use when calculating the stabiliser

`hash' & the dictionary info of `schreierTree'

`subGenerators' & the additional generators of the bigger group

`ssInfo' & main structure for the base and sgs of the smaller group

`SGS' & strong generating set for the smaller group

`identity' & the group identity

`points' & the point set to which `point' belong

`IsIdentity' & the identity check function for the larger group

`field' & the finite field of the larger group
\enditems




\>MATRIXSS_VerifySingleGenerator( generators, schreierTree, point, action, hash, subGenerator, ssInfo, SGS, identity ) F

Verifies that the stabiliser of the group generated by `generators', at the
point `point' is the same group as the group generated by `generators' minus
`subGenerator'. If so, the identity is returned, and otherwise an element
that is in the difference is returned.

`ssInfo' and `SGS' should be a base and strong generating set for the 
smaller group.

\beginitems    
`generators' & generators of the bigger group
  
`schreierTree' & Schreier tree for the orbit of `point' under `action' of 
                the group generated by `generators'
  
`point' & the point to get the stabiliser of

`action' & the action to use when calculating the stabiliser

`hash' & the dictionary info of `schreierTree'

`subGenerator' & the additional generator of the bigger group

`ssInfo' & main structure for the base and sgs of the smaller group

`SGS' & strong generating set for the smaller group

`identity' & the group identity
\enditems




\>MATRIXSS_StabiliserGens( ssInfo, partialSGS, point, action, dictinfo, identity ) F

Return generators of the stabiliser of the group generated by the strong
generators `partialSGS' at `point' under `action'. The `ssInfo' structure
should be a base for the group.

\beginitems        
`ssInfo' & main structure for the base
  
`partialSGS' & strong generating set for the given group

`point' & the point to get stabiliser at

`action' & the action to use when computing stabiliser

`dictinfo' & the dictionary info of `schreierTree'

`identity' & the group identity
\enditems




\>MATRIXSS_IsBlockOfImprimitivity( schreierTree, generators, block, action, identity ) F

Checks whether `block' is a block of imprimitivity for `action' of the group
given by `generators' on the set of points given by `schreierTree'.

\beginitems        
`schreierTree' & Schreier tree for the point set
  
`generators' & generators of the acting group 

`block' & the block to check for imprimitivity

`action' & the action to use

`identity' & the group identity
\enditems




\>MATRIXSS_DecomposeOrbit( schreierTree, root, generators, action, hash, identity ) F

Decompose the orbit given by `schreierTree', with root point `root', into
orbits of the group generated by `generators', under `action'.

\beginitems        
`schreierTree' & Schreier tree for the orbit to decompose
  
`root' & root point of `schreierTree'

`generators' & generators of the decomposing group 

`action' & the action of the decomposing group

`hash' & the dictionary info of `schreierTree'

`identity' & the group identity
\enditems




\>MATRIXSS_BaseChange( ssInfo, partialSGS, level, identity ) F

Flips the base point at `level' with the next higher base point and update
the `ssInfo' structure.

\beginitems        
`ssInfo' & main structure for the base
  
`partialSGS' & strong generating set corresponding to `ssInfo'

`level' & the level for the base change

`identity' & the group identity
\enditems




%%%%%%%%%%%%%%%%%%%%%%%%%%%%%%%%%%%%%%%%%%%%%%%%%%%%%%%%%%%%%%%%%%%%%%%%%%%%%%%
\Section{Test and benchmark routines}





These are the main routines for testing and benchmarking the package.




\>MatrixSchreierSimsTest( maxDegree, maxFieldSize ) F

Compares results of the above function with the built-in GAP Size method for
a bunch of classical matrix groups. (`GL', `SL', etc)
\beginitems
`maxDegree' & maximum matrix size for classical matrix groups to be used
           for testing

`maxFieldSize' & maximum finite field size for classical matrix groups to be
              used for testing
\enditems




\>MatrixSchreierSimsBenchmark( maxDegree, maxFieldSize, maxReeSize, maxSuzukiSize ) F

Check speed of package routines against classical matrix groups and the
matrix representations of Ree and Suzuki sporadic groups.
\beginitems
`maxDegree' & maximum matrix size for classical matrix groups to be used
           for testing

`maxFieldSize' & maximum finite field size for classical matrix groups to be
              used for testing

`maxReeSize' & maximum `ReeGroup' size, see "ref:Ree" in the reference 
              manual.

`maxSuzukiSize' & maximum `SuzukiGroup' size, see "ref:Sz" in the reference 
                 manual.
\enditems








These are auxiliary functions for test and benchmark.




\>MATRIXSS_GetTestGroups( maxDegree, maxFieldSize ) F

Creates a list of classical matrix groups to use when testing the package.
The groups are `GL', `SL', `GO', `SO', `GU' and `SU'.
\beginitems
`maxDegree' & maximum matrix size for classical matrix groups to be used
           for testing

`maxFieldSize' & maximum finite field size for classical matrix groups to be
              used for testing
\enditems




\>MATRIXSS_GetBenchmarkGroups( maxDegree, maxFieldSize ) F

Creates a list of classical matrix groups and sporadic groups to use when 
benchmarking the package.
The classical groups are `GL', `SL', `GO' and `SO'. The sporadic groups are
`Ree' and `Sz'.
\beginitems
`maxDegree' & maximum matrix size for classical matrix groups to be used
           for testing

`maxFieldSize' & maximum finite field size for classical matrix groups to be
              used for testing

`maxReeSize' & maximum `ReeGroup' size, see "ref:Ree" in the reference 
              manual.

`maxSuzukiSize' & maximum `SuzukiGroup' size, see "ref:Sz" in the reference 
                 manual.
\enditems




\>MATRIXSS_TimedCall( call, args ) F

Runs the specified function with arguments and return the running time in 
milliseconds, as given by Runtime, see "ref:Runtime" in the reference 
manual.
\beginitems
`call' & function to call

`args' & list of arguments to `call'
\enditems




%%%%%%%%%%%%%%%%%%%%%%%%%%%%%%%%%%%%%%%%%%%%%%%%%%%%%%%%%%%%%%%%%%%%%%%%%%%%%%%
%%
%E  code.msk  . . . . . . . . . . . . . . . . . . . . . . . . . . ends here
